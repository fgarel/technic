

%definition de quelques macros pour l'inseertion des photos

% 1er type d'insertion : on insert une image
% Les deux arguments sont :
%  - Nom de la photo
%  - Numero du point
%\includePhotoTiersVerticalDixQuatre{P1020538}{1009}


% 2d type d'insertion : on insert une image et son zoom
% il aurait été plus simple de faire cela dans une seule et même macro, mais cela n'a pas pu être possible

% nous avons donc plusiseurs commandes
% - pour l'insertion de photo horizontale et pour l'insertion de photo verticale
% - pour des zoom centré (huitSix) ou pour des zoom décalés


%%%%%%%%%%%%%%%%%%%%%%%%%%%%%%%%%%%%%%%%%%%%%%%%%%%%%%%%%
\newcommand\includePhotoTiersHorizontal[2]{% 3 photos par ligne, orientation paysage
    \begin{subfigure}[b]{0.3\textwidth}
        \includegraphics[angle=0,viewport=0 0 1600 1200,bb=0 0 4000 3000,width=4.4cm,keepaspectratio=true]{#1.JPG}
        \caption{Point #2}
        \label{#1}
    \end{subfigure}
}
\newcommand\includePhotoTiersVertical[2]{% 3 photos par ligne, orientation portrait
    \begin{subfigure}[b]{0.3\textwidth}
        \includegraphics[angle=-90,viewport=0 0 1600 1200,bb=0 0 4000 3000,width=3.3cm,keepaspectratio=true]{#1.JPG}
        \caption{Point #2}
        \label{#1}
    \end{subfigure}
}

\newcommand\includePhotoMoitieHorizontal[2]{% 2 photos par ligne, orientation paysage
    \begin{subfigure}[b]{0.45\textwidth}
        \includegraphics[angle=0,viewport=0 0 1600 1200,bb=0 0 4000 3000,width=6.6cm,keepaspectratio=true]{#1.JPG}
        \caption{Point #2}
        \label{#1}
    \end{subfigure}
}

% Vertical
\newcommand\includePhotoMoitieVertical[2]{% 2 photos par ligne, orientation portrait
    \begin{subfigure}[b]{0.45\textwidth}
        \includegraphics[angle=-90,viewport=0 0 1600 1200,bb=0 0 4000 3000,width=4.4cm,keepaspectratio=true]{#1.JPG}
        \caption{Point #2}
        \label{#1}
    \end{subfigure}
}

% vertical retourné (2V)
\newcommand\includePhotoMoitieVVertical[2]{% 2 photos par ligne, orientation portrait
    \begin{subfigure}[b]{0.45\textwidth}
        \includegraphics[angle=90,viewport=0 0 1600 1200,bb=0 0 4000 3000,width=4.4cm,keepaspectratio=true]{#1.JPG}
        \caption{Point #2}
        \label{#1}
    \end{subfigure}
}
%%%%%%%%%%%%%%%%%%%%%%%%%%%%%%%%%%%%%%%%%%%%%%%%%%%%%%%%%
% photo horizontale, avec un zoom au centre
% definition d'une nouvelle commande avec deux arguments
% le premier est la photo, le second est le point
\newcommand\includePhotoDetailHorizontalHuitSix[2]{
\begin{figure}[!h]
    \centering
    \includePhotoMoitieHorizontal{#1}{#2}
    %\begin{subfigure}[b]{0.45\textwidth}
    %    \includegraphics[angle=0,viewport=0 0 1600 1200,bb=0 0 4000 3000,width=6.6cm,keepaspectratio=true]{#1.JPG}
    %    \caption{Point #2}%+200+150
    %    \label{#1}
    %\end{subfigure}
    ~
    \begin{subfigure}[b]{0.45\textwidth}
        \includegraphics[angle=0,viewport=600 450 1000 750,bb=0 0 4000 3000,width=6.6cm,keepaspectratio=true,clip=true]{#1.JPG}
        \caption{Détail du point #2}%+200+150
        \label{Zoom #1}
    \end{subfigure}
    \caption{Détail du point #2}
    \label{Point Cible #2}
\end{figure}
}

% photo horizontale, avec un zoom décalé
\newcommand\includePhotoDetailHorizontalHuitQuatre[2]{% on centre sur le point de coordonnes 800 400
\begin{figure}[!h]
    \centering
    \includePhotoMoitieHorizontal{#1}{#2}
    %\begin{subfigure}[b]{0.45\textwidth}
    %    \includegraphics[angle=0,viewport=0 0 1600 1200,bb=0 0 4000 3000,width=6.6cm,keepaspectratio=true]{#1.JPG}
    %    \caption{Point #2}%+200+150
    %    \label{#1}
    %\end{subfigure}
    ~
    \begin{subfigure}[b]{0.45\textwidth}
        \includegraphics[angle=0,viewport=600 250 1000 550,bb=0 0 4000 3000,width=6.6cm,keepaspectratio=true,clip=true]{#1.JPG}
        \caption{Détail du point #2}%+200+150
        \label{Zoom #1}
    \end{subfigure}
    \caption{Détail du point #2}
    \label{Point Cible #2}
\end{figure}
}

% photo horizontale, avec un zoom décalé
\newcommand\includePhotoDetailHorizontalQuatreQuatre[2]{% on centre sur le point de coordonnes 400 400
\begin{figure}[!h]
    \centering
    \includePhotoMoitieHorizontal{#1}{#2}
    %\begin{subfigure}[b]{0.45\textwidth}
    %    \includegraphics[angle=0,viewport=0 0 1600 1200,bb=0 0 4000 3000,width=6.6cm,keepaspectratio=true]{#1.JPG}
    %    \caption{Point #2}%+200+150
    %    \label{#1}
    %\end{subfigure}
    ~
    \begin{subfigure}[b]{0.45\textwidth}
        \includegraphics[angle=0,viewport=200 250 600 550,bb=0 0 4000 3000,width=6.6cm,keepaspectratio=true,clip=true]{#1.JPG}
        \caption{Détail du point #2}%+200+150
        \label{Zoom #1}
    \end{subfigure}
    \caption{Détail du point #2}
    \label{Point Cible #2}
\end{figure}
}

% photo horizontale, avec un zoom décalé
\newcommand\includePhotoDetailHorizontalSixSix[2]{% on centre sur le point de coordonnes 600 600
\begin{figure}[!h]
    \centering
    \includePhotoMoitieHorizontal{#1}{#2}
    %\begin{subfigure}[b]{0.45\textwidth}
    %    \includegraphics[angle=0,viewport=0 0 1600 1200,bb=0 0 4000 3000,width=6.6cm,keepaspectratio=true]{#1.JPG}
    %    \caption{Point #2}%+200+150
    %    \label{#1}
    %\end{subfigure}
    ~
    \begin{subfigure}[b]{0.45\textwidth}
        \includegraphics[angle=0,viewport=400 450 800 750,bb=0 0 4000 3000,width=6.6cm,keepaspectratio=true,clip=true]{#1.JPG}
        \caption{Détail du point #2}%+200+150
        \label{Zoom #1}
    \end{subfigure}
    \caption{Détail du point #2}
    \label{Point Cible #2}
\end{figure}
}

% photo horizontale, avec un zoom décalé
\newcommand\includePhotoDetailHorizontalHuitHuit[2]{% on centre sur le point de coordonnes 800 800
\begin{figure}[!h]
    \centering
    \includePhotoMoitieHorizontal{#1}{#2}
    %\begin{subfigure}[b]{0.45\textwidth}
    %    \includegraphics[angle=0,viewport=0 0 1600 1200,bb=0 0 4000 3000,width=6.6cm,keepaspectratio=true]{#1.JPG}
    %    \caption{Point #2}%+200+150
    %    \label{#1}
    %\end{subfigure}
    ~
    \begin{subfigure}[b]{0.45\textwidth}
        \includegraphics[angle=0,viewport=600 650 1000 950,bb=0 0 4000 3000,width=6.6cm,keepaspectratio=true,clip=true]{#1.JPG}
        \caption{Détail du point #2}%+200+150
        \label{Zoom #1}
    \end{subfigure}
    \caption{Détail du point #2}
    \label{Point Cible #2}
\end{figure}
}

%%%%%%%%%%%%%%%%%%%%%%%%%%%%%%%%%%%%%%%%%%%%%%%%%%%%%%%%%
% photo verticale, avec un zoom au centre
\newcommand\includePhotoDetailVerticalHuitSix[2]{
\begin{figure}[!h]
    \centering
    \includePhotoMoitieVertical{#1}{#2}
    %\begin{subfigure}[b]{0.45\textwidth}
    %    \includegraphics[angle=-90,viewport=0 0 1600 1200,bb=0 0 4000 3000,width=4.4cm,keepaspectratio=true]{#1.JPG}
    %    \caption{Point #2}%+200+150
    %    \label{#1}
    %\end{subfigure}
    ~
    \begin{subfigure}[b]{0.45\textwidth}
        \includegraphics[angle=-90,viewport=600 450 1000 750,bb=0 0 4000 3000,width=4.4cm,keepaspectratio=true,clip=true]{#1.JPG}
        \caption{Détail du point #2}%+200+150
        \label{Zoom #1}
    \end{subfigure}
    \caption{Détail du point #2}
    \label{Point Cible #2}
\end{figure}
}

% photo verticale, avec un zoom décalé
%\newcommand\includePhotoDetailVerticalSixSix[2]{% On centre sur le point de cooreonnes 600 600
%\begin{figure}[!h]
%    \centering
%    \begin{subfigure}[b]{0.45\textwidth}
%        \includegraphics[angle=-90,viewport=0 0 1600 1200,bb=0 0 4000 3000,width=4.4cm,keepaspectratio=true]{#1.JPG}
%        \caption{Point #2}%+200+150
%        \label{#1}
%    \end{subfigure}
%    ~
%    \begin{subfigure}[b]{0.45\textwidth}
%        \includegraphics[angle=-90,viewport=400 450 800 750,bb=0 0 4000 3000,width=4.4cm,keepaspectratio=true,clip=true]{#1.JPG}
%        \caption{Détail du point #2}%+200+150
%        \label{Zoom #1}
%    \end{subfigure}
%    \caption{Détail du point #2}
%    \label{Point Cible #2}
%\end{figure}
%}


% photo verticale, avec un zoom décalé
\newcommand\includePhotoDetailVerticalDixQuatre[2]{% On centre sur le point de cooreonnes 1000 400
\begin{figure}[!h]
    \centering
    \includePhotoMoitieVertical{#1}{#2}
    %\begin{subfigure}[b]{0.45\textwidth}
    %    \includegraphics[angle=-90,viewport=0 0 1600 1200,bb=0 0 4000 3000,width=4.4cm,keepaspectratio=true]{#1.JPG}
    %    \caption{Point #2}%+200+150
    %    \label{#1}
    %\end{subfigure}
    ~
    \begin{subfigure}[b]{0.45\textwidth}
        \includegraphics[angle=-90,viewport=800 250 1200 550,bb=0 0 4000 3000,width=4.4cm,keepaspectratio=true,clip=true]{#1.JPG}
        \caption{Détail du point #2}%+200+150
        \label{Zoom #1}
    \end{subfigure}
    \caption{Détail du point #2}
    \label{Point Cible #2}
\end{figure}
}

%%%%%%%%%%%%%%%%%%%%%%%%%%%%%%%%%%%%%%%%%%%%%%%%%%%%%%%%%
% photo verticaleRetourné, avec un zoom au centre
\newcommand\includePhotoDetailVVerticalHuitSix[2]{
\begin{figure}[!h]
    \centering
    \includePhotoMoitieVVertical{#1}{#2}
    %\begin{subfigure}[b]{0.45\textwidth}
    %    \includegraphics[angle=-90,viewport=0 0 1600 1200,bb=0 0 4000 3000,width=4.4cm,keepaspectratio=true]{#1.JPG}
    %    \caption{Point #2}%+200+150
    %    \label{#1}
    %\end{subfigure}
    ~
    \begin{subfigure}[b]{0.45\textwidth}
        \includegraphics[angle=90,viewport=600 450 1000 750,bb=0 0 4000 3000,width=4.4cm,keepaspectratio=true,clip=true]{#1.JPG}
        \caption{Détail du point #2}%+200+150
        \label{Zoom #1}
    \end{subfigure}
    \caption{Détail du point #2}
    \label{Point Cible #2}
\end{figure}
}



% Pour info, les essais concernant les calculs artithmétiques
% paquet fp, mais par la suite,
% je n'arrive pas à faire passer mystr en tant qu'option valide à includegraphics
\newcommand{\offsetxy}[4]{
        \FPeval{\dxi}{600+#3}
        \FPclip{\dxi}{\dxi}
        \FPeval{\dxf}{1000+#3}
        \FPclip{\dxf}{\dxf}
        \FPeval{\dyi}{450+#4}
        \FPclip{\dyi}{\dyi}
        \FPeval{\dyf}{750+#4}
        \FPclip{\dyf}{\dyf}
        \newcommand\mystr{angle=0,viewport=\dxi \dyi \dxf \dyf,bb=0 0 4000 3000,width=6.6cm,keepaspectratio=true,clip=true}
        \mystr
        %\includegraphics[\mystr]{#1.JPG}
        \newcommand\viewport{\dxi \dyi \dxf \dyf}
        \viewport

        \includegraphics[angle=0,bb=0 0 4000 3000,width=6.6cm,keepaspectratio=true,clip=true]{#1.JPG}
        %\includegraphics[viewport=\viewport,angle=0,bb=0 0 4000 3000,width=6.6cm,keepaspectratio=true,clip=true]{#1.JPG}
}
%\offsetxy{P1020539}{1010}{000}{000}


