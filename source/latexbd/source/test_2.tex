\documentclass[12pt,titlepage,oneside]{book}

%\usepackage[latin1]{inputenc}
\usepackage[utf8]{inputenc}
\usepackage[T1]{fontenc}
\usepackage[french]{babel}

\usepackage[a3paper,landscape,vmargin=2cm,hmargin=1.5cm]{geometry}
\usepackage{multirow}
\usepackage{supertabular}
\usepackage{array}
\usepackage{lscape}

\usepackage{pslatex}
\usepackage{amsmath}
\usepackage{amsfonts}
\usepackage{amssymb}
\author{Direction des Systèmes d'Information\\
Pôle Géomatique}
\title{Catalogue des objets topographiques}

% Les parametres de connexion
\begin{lbdpython}
db = create_engine('postgres://fred@localhost/topographie')
metadata = BoundMetaData(db)
\end{lbdpython}


% les requetes
\begin{lbdpython}
# on definit les tables : cf syntaxe SQLAlchemy :
# paragraphe creating a table
theme_table = Table('theme', metadata, autoload=True)
domaine_table = Table('domaine', metadata, autoload=True)
classe_table = Table('classe', metadata, autoload=True)
objet_table = Table('objet', metadata, autoload=True)

# on definit les alias
th = theme_table.alias(th)
dom = domaine_table.alias()
cls = classe_table.alias()
obj = objet_table.alias()

# on initialise les requetes
#ReqThS = {}
req_domaine_select = {}
req_classe_select = {}
req_objet_select = {}
req_theme_count = {}
req_domaine_count = {}
req_classe_count = {}
req_objet_count = {}

# on definit les requetes selection
#ReqThS = select([th.c.theme_id, th.c.theme_libelle])
ReqThS = select([theme_table.c.theme_id, theme_table.c.theme_libelle])

# on definit les requetes specifiques au comptage
# ces requetes sont utilisés pour la mise en forme du tableau
# -- pour chaque theme, calcul du nombre de domaines
#SELECT
#   theme.theme_id as ThemeId,
#   count(1)
#FROM theme, domaine
#WHERE theme.theme_id = domaine.theme_id 
#GROUP BY theme.theme_id
#ORDER BY theme.theme_id;
req_theme_count = select([th.c.theme_id, count(1)]),
                  and_(th.c.theme_id == dom.c.theme_id),
                  group_by=[th.c.theme_id]),
                  order_by=[th.c.theme_id])
\end{lbdpython}

\begin{document}
\maketitle{}
% essai code python pour fabriquer un tableau
\begin{lbdpython}
but.write(r"\begin{tabular}{|m{3cm}|m{3cm}|c|m{3cm}|}")
but.write("\n")
but.write(r"\hline")
but.write("\n")
for i in range(1,12,1):
  # le r à l'interieur des parentheses empeche l'interpretation ?
  but.write(r"1 & 2 & 3 & 4 \\\hline")
  but.write("\n")
but.write(r"\end{tabular}")
but.write("\n")
\end{lbdpython}

% essai code python pour parcourir resultat requete
\begin{lbdpython}
but.write(r"avant \\")
but.write("\n")
but.write(r"\begin{tabular}{|m{3cm}|m{3cm}|}")
but.write("\n")
but.write(r"\hline")
but.write("\n")
result = ReqThS.execute()
for row in result:
  chaine = r"%s & %s \\" %(row.theme_id,row.theme_libelle)
  but.write(chaine)
  but.write(r"\hline")
  but.write("\n")
result = req_theme_count.execute()
for row in result:
  chaine = r"%s & %s \\" %(row.theme_id,row.count)
  but.write(chaine)
  but.write(r"\hline")
  but.write("\n")

but.write(r"\end{tabular}")
but.write("\n")
but.write(r"apres \\")
but.write("\n")
\end{lbdpython}

% essai d'un tableau
\begin{tabular}{|m{3cm}|m{3cm}|c|m{3cm}|}
\hline
\multicolumn{4}{|c|}{Essai Tableau} \\
\hline
\multicolumn{1}{|c|}{Theme} & \multicolumn{1}{|c|}{Domaine}  & \multicolumn{2}{|c|}{classe} \\
\cline{1-4}      \multirow{5}*{theme\_01} & \multirow{2}*{domaine\_01} & 1 & 2 \\
\cline{3-4}                               &                            & 2 & 2 \\
\cline{2-4}                               & \multirow{3}*{domaine\_02} & 3 & 2 \\
\cline{3-4}                               &                            & 4 & 2 \\
\cline{3-4}                               &                            & 5 & 2 \\
\cline{1-4}      \multirow{6}*{theme\_02} & \multirow{3}*{domaine\_03} & 6 & 2 \\
\cline{3-4}                               &                            & 7 & 2 \\
\cline{3-4}                               &                            & 8 & 2 \\
\cline{2-4}                               & \multirow{3}*{domaine\_04} & 9 & 2 \\
\cline{3-4}                               &                            & 10 & 2 \\
\cline{3-4}                               &                            & 11 & 2 \\
\cline{1-4}      \multirow{2}*{theme\_03} & \multirow{1}*{domaine\_05} & 12 & 2 \\
\cline{2-4}                               & \multirow{1}*{domaine\_06} & 13 & 2 \\
\cline{1-4}      \multirow{1}*{theme\_04} & \multirow{1}*{domaine\_07} & 14 & 2 \\
\hline
\end{tabular}



\end{document}