\documentclass[12pt,titlepage,oneside]{book}

%\usepackage[latin1]{inputenc}
\usepackage[utf8]{inputenc}
\usepackage[T1]{fontenc}
\usepackage[french]{babel}

\usepackage[a3paper,landscape,vmargin=2cm,hmargin=1.5cm]{geometry}
\usepackage{multirow}
\usepackage{supertabular}
\usepackage{array}
\usepackage{lscape}
\usepackage{graphicx}

\usepackage{pslatex}
\usepackage{amsmath}
\usepackage{amsfonts}
\usepackage{amssymb}
\author{Direction des Systèmes d'Information\\
Pôle Géomatique}
\title{Catalogue des objets topographiques}

% Les parametres de connexion
\begin{lbdpython}
# #################### #
# debut du code python #
# #################### #
# connexion à la base topographie
# -------------------------------
db = create_engine('postgres://fred@localhost/topographie')
metadata = BoundMetaData(db)


# definition des requetes : cf syntaxe SQLAlchemy
# -----------------------------------------------
# on definit les tables
theme_table = Table('theme', metadata, autoload=True)
domaine_table = Table('domaine', metadata, autoload=True)
classe_table = Table('classe', metadata, autoload=True)
objet_table = Table('objet', metadata, autoload=True)
initiale_table = Table('representation_initiale', metadata, autoload=True)
chagneau_table = Table('rep_chagneau', metadata, autoload=True)
objet_attribut_table = Table('objet_attribut_topo', metadata, autoload=True)


# on definit les alias
th = theme_table.alias('th')
dom = domaine_table.alias('dom')
cls = classe_table.alias('cls')
obj = objet_table.alias('obj')
ini = initiale_table.alias('ini')
cha = chagneau_table.alias('cha')
att = objet_attribut_table.alias('att')

# les requetes seront definies et executées au fur et à mesure

\end{lbdpython}


\begin{document}


\chapter{Catalogue des objets topographiques non-pris en charge dans le SIG}

% Les objets topo
% ^^^^^^^^^^^^^^^
\section{Les Objets - Présentation sous la forme d'un document structuré}

\vspace{\baselineskip}

% les options de supertabular sont a definir
% avant \begin{supertabular}
\tablefirsthead{
%\hline
%\multicolumn{3}{|c|}{Objet} \\
%\hline
%\multicolumn{1}{|c|}{Theme / Domaine / Classe} & \multicolumn{2}{|c|}{Objet} \\
\hline
Classe & Identifiant & Objet & Infos\\
\hline
}
\tablehead{
%\hline
%\multicolumn{3}{|c|}{Objet} \\
%\hline
%\multicolumn{1}{|c|}{Theme / Domaine / Classe} & \multicolumn{2}{|c|}{Objet} \\
\hline
Classe & Identifiant & Objet & Infos\\
\hline
}
%\begin{supertabular}{|p{10cm}|c|p{15cm}|p{10cm}}



\begin{lbdpython}

# on definit un booleen permettant de distinguer
# la permiere ligne des autres lignes
# c'est la premiere ligne pour un niveau 0
# il faudra ecrire begin{supertabular}
debut_niveau_00 = 1


# on definit une requete de comptage
# cette requete est utilisée pour la mise en forme du tableau
#-- pour chaque theme, calcul du nombre d'objets
#SELECT
#   theme.theme_id as ThemeId,
#   count(1)
#FROM theme, domaine, classe, objet
#WHERE theme.theme_id = domaine.theme_id 
#  AND domaine.domaine_id = classe.domaine_id
#  AND classe.classe_id = objet.classe_id
#GROUP BY theme.theme_id
#ORDER BY theme.theme_id;

req_theme_count = select([th.c.theme_id, func.count(1)],
                  and_(th.c.theme_id == dom.c.theme_id,
                       dom.c.domaine_id == cls.c.domaine_id,
                       cls.c.classe_id == obj.c.classe_id,
                       obj.c.bool_objet_sig == 0),
                  group_by=[th.c.theme_id],
                  order_by=[th.c.theme_id])
result_niveau_01 = req_theme_count.execute()
for row_niveau_01 in result_niveau_01:
   # on definit un booleen permettant de distinguer dans le tableau
   # la permiere ligne des autres lignes
   # c'est la premiere ligne pour un niveau 1
   debut_niveau_01 = 1
   theme_encours_id = row_niveau_01.theme_id
   # on liste les domaines et on compte les objets correspondant à ce theme_encours_id
   #-- pour chaque domaine, calcul du nombre d'objets
   #SELECT
   #   theme.theme_id as ThemeId,
   #   domaine.domaine_id as DomaineId,
   #   count(1)
   #FROM theme, domaine, classe, objet
   #WHERE theme.theme_id = domaine.theme_id
   #  AND domaine.domaine_id = classe.domaine_id
   #  AND classe.classe_id = objet.classe_id
   #GROUP BY theme.theme_id, domaine.domaine_id
   #ORDER BY theme.theme_id, domaine.domaine_id;
   req_domaine_count = select([th.c.theme_id, dom.c.domaine_id, func.count(1)],
                       and_(th.c.theme_id == dom.c.theme_id,
                            dom.c.domaine_id == cls.c.domaine_id,
                            cls.c.classe_id == obj.c.classe_id,
                            th.c.theme_id == theme_encours_id,
                            obj.c.bool_objet_sig == 0),
                       group_by=[th.c.theme_id, dom.c.domaine_id],
                       order_by=[th.c.theme_id, dom.c.domaine_id])
   result_niveau_02 = req_domaine_count.execute()
   for row_niveau_02 in result_niveau_02:
      # on definit un booleen permettant de distinguer dans le tableau
      # la permiere ligne des autres lignes
      # c'est la premiere ligne pour un niveau 2
      debut_niveau_02 = 1
      domaine_encours_id = row_niveau_02.domaine_id
      # on liste les classes et on compte les objets correspondant à ce domaine_encours_id
      #-- pour chaque classe, calcul du nombre d'objets
      #SELECT
      #   theme.theme_id as ThemeId,
      #   domaine.domaine_id as DomaineId,
      #   classe.classe_id as Classe_id,
      #   count(1)
      #FROM theme, domaine, classe, objet
      #WHERE theme.theme_id = domaine.theme_id 
      #  AND domaine.domaine_id = classe.domaine_id
      #  AND classe.classe_id = objet.classe_id
      #GROUP BY theme.theme_id, domaine.domaine_id, classe.classe_id
      #ORDER BY theme.theme_id, domaine.domaine_id, classe.classe_id;
      req_classe_count = select([th.c.theme_id, dom.c.domaine_id,
                                 cls.c.classe_id, func.count(1)],
                         and_(th.c.theme_id == dom.c.theme_id,
                              dom.c.domaine_id == cls.c.domaine_id,
                              cls.c.classe_id == obj.c.classe_id,
                              dom.c.domaine_id == domaine_encours_id,
                              obj.c.bool_objet_sig == 0),
                         group_by=[th.c.theme_id, dom.c.domaine_id, cls.c.classe_id],
                         order_by=[th.c.theme_id, dom.c.domaine_id, cls.c.classe_id])
      result_niveau_03 = req_classe_count.execute()
      for row_niveau_03 in result_niveau_03:
         # on definit un booleen permettant de distinguer dans le tableau
         # la permiere ligne des autres lignes
         # c'est la premiere ligne pour un niveau 3
         debut_niveau_03 = 1
         classe_encours_id = row_niveau_03.classe_id
         # recherche des classes correspondant à ce domaine_encours_id
         # -- selection objet
         #SELECT
         #   theme.theme_id as ThemeId,
         #   theme.theme_libelle as ThemeLibelle,
         #   domaine.domaine_id as DomaineId,
         #   domaine.domaine_libelle as DomaineLibelle,
         #   classe.classe_id as ClasseId,
         #   classe.classe_libelle as ClasseLibelle,
         #   objet.objet_id as ObjetId,
         #   objet.objet_libelle as ObjetLibelle
         #FROM theme, domaine, classe, objet
         #WHERE theme.theme_id = domaine.theme_id
         #  AND domaine.domaine_id = classe.domaine_id
         #  AND domaine.domaine_id = domaine_encours_id
         #ORDER by theme.theme_id, domaine.domaine_id, classe_id;
         req_objet_select = select([th.c.theme_id, th.c.theme_libelle,
                                    dom.c.domaine_id, dom.c.domaine_libelle,
                                    cls.c.classe_id, cls.c.classe_libelle,
                                    obj.c.objet_id, obj.c.objet_libelle],
                              and_(th.c.theme_id == dom.c.theme_id,
                                   dom.c.domaine_id == cls.c.domaine_id,                              
                                   cls.c.classe_id == obj.c.classe_id,                              
                                   cls.c.classe_id == classe_encours_id,
                                   obj.c.bool_objet_sig == 0),
                              order_by=[th.c.theme_id, dom.c.domaine_id, cls.c.classe_id, obj.c.objet_id])
         result_niveau_04 = req_objet_select.execute()
         for row_niveau_04 in result_niveau_04:
            if (debut_niveau_00 == 0) and (debut_niveau_01 == 1 or debut_niveau_02 == 1 or debut_niveau_03 == 1):
               but.write(r"\hline"+"\n")
               chaine = r"\end{supertabular}"
               but.write(chaine+"\n")
               # selection dans la table rep_chagneau des representations pour lles objets de la classe precedente
               cha_select = select([cha.c.code, cha.c.basename, cha.c.fullname],
                               and_(cha.c.code == obj.c.objet_id,
                                    cls.c.classe_id == obj.c.classe_id,
                                    cls.c.classe_id == classe_precedente_id),
                               group_by=[cha.c.code, cha.c.basename, cha.c.fullname],
                               order_by=[cha.c.code])
               result_cha_select = cha_select.execute()
               but.write(r"\begin{figure}[h!]"+"\n")
               but.write(r"  \hfill         % permet de caler à droite de la page"+"\n")
               for row_niveau_05 in result_cha_select:
                  but.write(r"  \begin{minipage}[t]{3cm}"+"\n")
                  but.write(r"    \begin{center}"+"\n")
                  chaine = row_niveau_05.fullname
                  but.write(r"      \includegraphics[height=1.3cm,width=3cm,clip=false,keepaspectratio=true]{" + chaine + "}"+"\n")
                  description_longue = row_niveau_05.basename
                  description_courte = row_niveau_05.code
                  but.write(r"      \caption[~" + description_courte + "]" 
                            + r"{\small{" + description_courte + r"~:} \tiny{" + description_longue + "}}"
                            + r"\label{" + description_longue + "}" + "\n")
                  but.write(r"    \end{center}"+"\n")
                  but.write(r"  \end{minipage}"+"\n")
               but.write(r"\end{figure}"+"\n") 

            if debut_niveau_01 == 1:
               chaine = r"\subsection{%s}" %row_niveau_04.theme_libelle
               but.write(chaine+"\n")
            else:
               but.write(""+"\n")
            if debut_niveau_02 == 1:
               chaine = r"\subsubsection{\large %s}" %row_niveau_04.domaine_libelle
               but.write(chaine+"\n")
            else:
               but.write(""+"\n")
            if debut_niveau_03 == 1:
               if row_niveau_03.count == 1:
                  chaine_colonne_03 = r"%s" %(row_niveau_04.classe_libelle)
               else:
                  chaine_colonne_03 = r"\multirow{%s}{10cm}{%s}" %(row_niveau_03.count, row_niveau_04.classe_libelle)
               chaine = r"\paragraph{%s}" %row_niveau_04.classe_libelle
               but.write(chaine+"\n")
               chaine = r"\noindent"
               but.write(chaine + "\n")
               chaine = r"\vspace{\baselineskip}"
               but.write(chaine + "\n" + "\n")
               chaine = r"\renewcommand{\arraystretch}{1.2}"
               but.write(chaine + "\n")
               chaine = r"\begin{supertabular}{|p{10cm}|c|p{13cm}|p{12cm}|}"
               but.write(chaine+"\n")
               classe_encours_id = row_niveau_03.classe_id
               classe_precedente_id = row_niveau_03.classe_id
            else:
               chaine_colonne_03 = r"                  "
            chaine_cline = r"\cline{2-4}"
            objet_encours_id = row_niveau_04.objet_id
            debut_niveau_00 = 0
            debut_niveau_01 = 0
            debut_niveau_02 = 0
            debut_niveau_03 = 0
            chaine_colonne_04 = r"%s" %(row_niveau_04.objet_id)
            chaine_colonne_05 = r"%s" %(row_niveau_04.objet_libelle)
            chaine_colonne_06 = r"Liste des attributs :" + "\n"
            chaine_colonne_06 += r"\begin{enumerate}" + "\n"
            # selection dans la table objet_attribut_topo des attributs pour les objets de la classe precedente
            att_select = select([att.c.code_objet, att.c.code_attribut, att.c.libelle_attribut],
                         and_(att.c.code_objet == objet_encours_id),
                         group_by=[att.c.code_objet, att.c.code_attribut, att.c.libelle_attribut],
                         order_by=[att.c.libelle_attribut])
            result_att_select = att_select.execute()
            for row_niveau_06 in result_att_select:  
                chaine_colonne_06 +=  r"  \item %s" %row_niveau_06.libelle_attribut
            chaine_colonne_06 += r"\end{enumerate}"+"\n"

            but.write(chaine_cline + " " + chaine_colonne_03 
                      + " & " + chaine_colonne_04
                      + " & " + chaine_colonne_05
                      + " & " + chaine_colonne_06
                      + r"\\" + "\n")
# on n'oublie pas de refermer le tableau
but.write(r"\hline"+"\n")
chaine = r"\end{supertabular}"
but.write(chaine+"\n")
# pour le dernier element de la liste, ne pas oublier de faire les figures....
# selection dans la table rep_chagneau des representations pour les objets de la classe precedente
cha_select = select([cha.c.code, cha.c.basename, cha.c.fullname],
             and_(cha.c.code == obj.c.objet_id,
                  cls.c.classe_id == obj.c.classe_id,
                  cls.c.classe_id == classe_precedente_id),
             group_by=[cha.c.code, cha.c.basename, cha.c.fullname],
             order_by=[cha.c.code])
result_cha_select = cha_select.execute()
but.write(r"\begin{figure}[h!]"+"\n")
but.write(r"  \hfill         % permet de caler à droite de la page"+"\n")
for row_niveau_05 in result_cha_select:
   but.write(r"  \begin{minipage}[t]{3cm}"+"\n")
   but.write(r"    \begin{center}"+"\n")
   chaine = row_niveau_05.fullname
   but.write(r"      \includegraphics[height=1.3cm,width=3cm,clip=false,keepaspectratio=true]{" + chaine + "}"+"\n")
   description_longue = row_niveau_05.basename
   description_courte = row_niveau_05.code
   but.write(r"      \caption[~" + description_courte + "]" 
             + r"{\small{" + description_courte + r"~:} \tiny{" + description_longue + "}}"
             + r"\label{" + description_longue + "}" + "\n")
   but.write(r"    \end{center}"+"\n")
   but.write(r"  \end{minipage}"+"\n")
but.write(r"\end{figure}"+"\n") 

\end{lbdpython}




\listoffigures

\end{document}