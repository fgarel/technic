\documentclass[12pt,titlepage,oneside]{book}

%\usepackage[latin1]{inputenc}
\usepackage[utf8]{inputenc}
\usepackage[T1]{fontenc}
\usepackage[french]{babel}

\usepackage[a3paper,landscape,vmargin=2cm,hmargin=1.5cm]{geometry}
\usepackage{multirow}
\usepackage{supertabular}
\usepackage{array}
\usepackage{lscape}

\usepackage{pslatex}
\usepackage{amsmath}
\usepackage{amsfonts}
\usepackage{amssymb}
\author{Direction des Systèmes d'Information\\
Pôle Géomatique}
\title{Catalogue des objets topographiques}

% Les parametres de connexion

\begin{document}

%\begin{landscape}
\maketitle{}
%\end{landscape}


% dans le frontmatter, on met la préface et l'introduction generale
% dans le mainmatter, on met les differents chapitres
% dans le backmatter, on met la conclusion general et le sommaire
\frontmatter

% préface
% ^^^^^^^
\chapter{Préface}
Les services techniques de la ville de La Rochelle sont amenés à manipuler des plans à grande échelle.
Ils peuvent parfois aussi être amenés à gérer ces objets topographiques qui sont représentés sur ces plans.
En effet, certains de ces objets peuvent être sous la responsabilité de la ville de La Rochelle. 

\vspace{\baselineskip}
Le but de ce document est de recenser ces objets topographiques et de proposer une organisation informatique facilitant la gestion de ces objets.

\vspace{\baselineskip}
Dans ce document, on appelle objet topographique tout objet pouvant être réprésenté sur un plan à grande échelle.

% introdution generale
% ^^^^^^^^^^^^^^^^^^^^
\chapter{Introduction}
Les objets topographiques ont été classés.
Ce classement se décompose en Thèmes, Domaines et Classes.

Dans un premier temps, on présentera ce système de classement, puis, on listera ces différents objets en fonction de leur prise en charge actuel dans le SIG :
\begin{enumerate}
  \item l'objet n'est pas actuellement présent dans le SIG,
  \item l'objet est déjà pris en charge dans le SIG,
  \item liste de tous les objets, quelque soit son niveau de prise en charge.
\end{enumerate}

\mainmatter

\chapter{Les objets SIG}



% Les objets SIG
% ^^^^^^^^^^^^^^
\section{Les Objets - Présentation sous la forme d'un document structuré}

\vspace{\baselineskip}
% Requete req\_objet
%%\texdbdef{##req\_objet}{
%%SELECT
%%   theme.theme\_id as ThemeId,
%%   theme.theme\_libelle as ThemeLibelle,
%%   domaine.domaine\_id as DomaineId,
%%   domaine.domaine\_libelle as DomaineLibelle,
%%   classe.classe\_id as ClasseId,
%%   classe.classe\_libelle as ClasseLibelle,
%%   objet.objet\_id as ObjetId,
%%   objet.objet\_libelle as ObjetLibelle
%%FROM theme, domaine, classe, objet
%%WHERE theme.theme\_id = domaine.theme\_id
%%  AND domaine.domaine\_id = classe.domaine\_id
%%  AND classe.classe\_id = objet.classe\_id
%%ORDER by theme.theme\_id, domaine.domaine\_id, classe.classe\_id, objet.objet\_id; }

% les options de supertabular sont a definir
% avant \begin{supertabular}
\tablefirsthead{
%\hline
%\multicolumn{3}{|c|}{Objet} \\
%\hline
%\multicolumn{1}{|c|}{Theme / Domaine / Classe} & \multicolumn{2}{|c|}{Objet} \\
\hline
Classe & Identifiant & Objet \\
\hline
}
\tablehead{
%\hline
%\multicolumn{3}{|c|}{Objet} \\
%\hline
%\multicolumn{1}{|c|}{Theme / Domaine / Classe} & \multicolumn{2}{|c|}{Objet} \\
\hline
Classe & Identifiant & Objet \\
\hline
}
%\begin{supertabular}{|m{10cm}|c|m{15cm}|}

\subsection{Raster, Prise de vue}
\subsubsection{\large Autres prises de vue}
\paragraph{Autres prises de vue}
\noindent
\vspace{\baselineskip}

\renewcommand{\arraystretch}{1.2}
\begin{supertabular}{|p{10cm}|c|p{15cm}|}
\cline{2-3} Autres prises de vue & 01\_03\_01\_01 & Autres prises de vue\\
\hline
\end{supertabular}
\subsection{Urbanisme et cartes à petite echelle}
\subsubsection{\large Limites administratives et cartes à petites echelle}
\paragraph{Échelle communale}
\noindent
\vspace{\baselineskip}

\renewcommand{\arraystretch}{1.2}
\begin{supertabular}{|p{10cm}|c|p{15cm}|}
\cline{2-3} Échelle communale & 03\_01\_03\_06 & Secteur Elu\\
\hline
\end{supertabular}

\subsubsection{\large Document d urbanisme}
\paragraph{Schéma Directeur ou SCOT}
\noindent
\vspace{\baselineskip}

\renewcommand{\arraystretch}{1.2}
\begin{supertabular}{|p{10cm}|c|p{15cm}|}
\cline{2-3} Schéma Directeur ou SCOT & 03\_02\_01\_01 & Schéma Directeur ou SCOT\\
\hline
\end{supertabular}


\paragraph{Prescriptions, Servitudes et autres zonages}
\noindent
\vspace{\baselineskip}

\renewcommand{\arraystretch}{1.2}
\begin{supertabular}{|p{10cm}|c|p{15cm}|}
\cline{2-3} \multirow{9}{10cm}{Prescriptions, Servitudes et autres zonages} & 03\_02\_03\_01 & Espace boisé classé\\


\cline{2-3}                    & 03\_02\_03\_02 & Emplacement réservé\\


\cline{2-3}                    & 03\_02\_03\_03 & Secteur Plan Masse\\


\cline{2-3}                    & 03\_02\_03\_04 & Zone de Publicité Restreinte\\


\cline{2-3}                    & 03\_02\_03\_05 & Emprise de construction\\


\cline{2-3}                    & 03\_02\_03\_06 & Perimetre de préemption (DPU)\\


\cline{2-3}                    & 03\_02\_03\_07 & Prescriptions se superposant aux zonage\\


\cline{2-3}                    & 03\_02\_03\_08 & Annexes du POS : Servitudes d Utilité Publique\\


\cline{2-3}                    & 03\_02\_03\_09 & Annexes du POS : Informations utiles\\
\hline
\end{supertabular}


\paragraph{ZPPAUP}
\noindent
\vspace{\baselineskip}

\renewcommand{\arraystretch}{1.2}
\begin{supertabular}{|p{10cm}|c|p{15cm}|}
\cline{2-3} ZPPAUP & 03\_02\_05\_01 & Zone de Protection du Patrimoine Architectural Urbain et Paysager\\
\hline
\end{supertabular}


\paragraph{Plan de sauvegarde et de mise en valeur}
\noindent
\vspace{\baselineskip}

\renewcommand{\arraystretch}{1.2}
\begin{supertabular}{|p{10cm}|c|p{15cm}|}
\cline{2-3} Plan de sauvegarde et de mise en valeur & 03\_02\_06\_01 & Plan de Sauvegarde et de Mise en Valeur\\
\hline
\end{supertabular}
\subsection{Réseau de Transport de personnes / déplacement}
\subsubsection{\large Referentiel Voirie filaire / Adresse}
\paragraph{Surfacique}
\noindent
\vspace{\baselineskip}

\renewcommand{\arraystretch}{1.2}
\begin{supertabular}{|p{10cm}|c|p{15cm}|}
\cline{2-3} Surfacique & 05\_01\_01\_01 & Objet Surfacique\\
\hline
\end{supertabular}


\paragraph{Carrefour}
\noindent
\vspace{\baselineskip}

\renewcommand{\arraystretch}{1.2}
\begin{supertabular}{|p{10cm}|c|p{15cm}|}
\cline{2-3} Carrefour & 05\_01\_03\_01 & Carrefour\\
\hline
\end{supertabular}

\subsubsection{\large Transport en commun}
\paragraph{Station transport en commun}
\noindent
\vspace{\baselineskip}

\renewcommand{\arraystretch}{1.2}
\begin{supertabular}{|p{10cm}|c|p{15cm}|}
\cline{2-3} Station transport en commun & 05\_02\_01\_01 & Station transport en commun\\
\hline
\end{supertabular}


\paragraph{Ligne transport en commun}
\noindent
\vspace{\baselineskip}

\renewcommand{\arraystretch}{1.2}
\begin{supertabular}{|p{10cm}|c|p{15cm}|}
\cline{2-3} Ligne transport en commun & 05\_02\_02\_01 & Ligne transport en commun\\
\hline
\end{supertabular}


\paragraph{Voie réservée taxi et transport en commun}
\noindent
\vspace{\baselineskip}

\renewcommand{\arraystretch}{1.2}
\begin{supertabular}{|p{10cm}|c|p{15cm}|}
\cline{2-3} Voie réservée taxi et transport en commun & 05\_02\_03\_01 & Voie réservée taxi et transport en commun\\
\hline
\end{supertabular}

\subsubsection{\large Cheminement pieton et PMR}
\paragraph{Voie piétonne}
\noindent
\vspace{\baselineskip}

\renewcommand{\arraystretch}{1.2}
\begin{supertabular}{|p{10cm}|c|p{15cm}|}
\cline{2-3} Voie piétonne & 05\_04\_01\_01 & Voie piétonne\\
\hline
\end{supertabular}


\paragraph{Passage souterrain}
\noindent
\vspace{\baselineskip}

\renewcommand{\arraystretch}{1.2}
\begin{supertabular}{|p{10cm}|c|p{15cm}|}
\cline{2-3} Passage souterrain & 05\_04\_02\_01 & Passage souterrain\\
\hline
\end{supertabular}


\paragraph{Pont, passerelle piétonne}
\noindent
\vspace{\baselineskip}

\renewcommand{\arraystretch}{1.2}
\begin{supertabular}{|p{10cm}|c|p{15cm}|}
\cline{2-3} Pont, passerelle piétonne & 05\_04\_03\_01 & Pont, passerelle piétonne\\
\hline
\end{supertabular}
\subsection{Voirie}
\subsubsection{\large Classement / gestion}
\paragraph{Gestion Conseil Général}
\noindent
\vspace{\baselineskip}

\renewcommand{\arraystretch}{1.2}
\begin{supertabular}{|p{10cm}|c|p{15cm}|}
\cline{2-3} Gestion Conseil Général & 07\_01\_01\_01 & Gestion Conseil Général\\
\hline
\end{supertabular}


\paragraph{Gestion CDA}
\noindent
\vspace{\baselineskip}

\renewcommand{\arraystretch}{1.2}
\begin{supertabular}{|p{10cm}|c|p{15cm}|}
\cline{2-3} Gestion CDA & 07\_01\_02\_01 & Gestion CDA\\
\hline
\end{supertabular}


\paragraph{Gestion Ville}
\noindent
\vspace{\baselineskip}

\renewcommand{\arraystretch}{1.2}
\begin{supertabular}{|p{10cm}|c|p{15cm}|}
\cline{2-3} Gestion Ville & 07\_01\_03\_01 & Gestion Ville\\
\hline
\end{supertabular}

\subsubsection{\large Chaussée / accotement}
\paragraph{Objets surfaciques (chaussée, accotement, revêtement,…)}
\noindent
\vspace{\baselineskip}

\renewcommand{\arraystretch}{1.2}
\begin{supertabular}{|p{10cm}|c|p{15cm}|}
\cline{2-3} \multirow{5}{10cm}{Objets surfaciques (chaussée, accotement, revêtement,…)} & 07\_02\_01\_01 & Chaussée (objet surfacique)\\


\cline{2-3}                    & 07\_02\_01\_02 & Revêtement (objet surfacique)\\


\cline{2-3}                    & 07\_02\_01\_03 & Terre Plein\\


\cline{2-3}                    & 07\_02\_01\_04 & Ralentisseur / Coussins berlinois\\


\cline{2-3}                    & 07\_02\_01\_05 & Capteur / Boucle de comptage\\
\hline
\end{supertabular}


\paragraph{Objets linéaires (limite de chaussée, limite de revêtement, bordure, fossé,…)}
\noindent
\vspace{\baselineskip}

\renewcommand{\arraystretch}{1.2}
\begin{supertabular}{|p{10cm}|c|p{15cm}|}
\cline{2-3} \multirow{8}{10cm}{Objets linéaires (limite de chaussée, limite de revêtement, bordure, fossé,…)} & 07\_02\_02\_01 & Limite de chaussée (objet linéaire)\\


\cline{2-3}                    & 07\_02\_02\_02 & Limite de revêtement (objet linéaire)\\


\cline{2-3}                    & 07\_02\_02\_03 & Fossé\\


\cline{2-3}                    & 07\_02\_02\_04 & Fil d eau / Écoulement pluvial\\


\cline{2-3}                    & 07\_02\_02\_05 & Caniveau\\


\cline{2-3}                    & 07\_02\_02\_06 & Gargouille\\


\cline{2-3}                    & 07\_02\_02\_07 & Bordure\\


\cline{2-3}                    & 07\_02\_02\_08 & Voie ferrée\\
\hline
\end{supertabular}


\paragraph{Objets ponctuels (...)}
\noindent
\vspace{\baselineskip}

\renewcommand{\arraystretch}{1.2}
\begin{supertabular}{|p{10cm}|c|p{15cm}|}
\cline{2-3} Objets ponctuels (...) & 07\_02\_03\_00 & Objet ponctuel Voirie\\
\hline
\end{supertabular}
\subsection{Bâti}
\subsubsection{\large Objet surfacique bati}
\paragraph{Emprise bâtiment}
\noindent
\vspace{\baselineskip}

\renewcommand{\arraystretch}{1.2}
\begin{supertabular}{|p{10cm}|c|p{15cm}|}
\cline{2-3} Emprise bâtiment & 09\_01\_01\_01 & Emprise bâtiment\\
\hline
\end{supertabular}


\paragraph{Ouvrage d art – pont – passerelle}
\noindent
\vspace{\baselineskip}

\renewcommand{\arraystretch}{1.2}
\begin{supertabular}{|p{10cm}|c|p{15cm}|}
\cline{2-3} Ouvrage d art – pont – passerelle & 09\_01\_02\_01 & Ouvrage d art – pont – passerelle\\
\hline
\end{supertabular}


\paragraph{Bloc béton – pilier}
\noindent
\vspace{\baselineskip}

\renewcommand{\arraystretch}{1.2}
\begin{supertabular}{|p{10cm}|c|p{15cm}|}
\cline{2-3} \multirow{3}{10cm}{Bloc béton – pilier} & 09\_01\_03\_01 & Bloc béton pour endiguement\\


\cline{2-3}                    & 09\_01\_03\_02 & Pilier\\


\cline{2-3}                    & 09\_01\_03\_03 & Soubassement\\
\hline
\end{supertabular}


\paragraph{Statue ou monument}
\noindent
\vspace{\baselineskip}

\renewcommand{\arraystretch}{1.2}
\begin{supertabular}{|p{10cm}|c|p{15cm}|}
\cline{2-3} Statue ou monument & 09\_01\_04\_01 & Statue ou monument\\
\hline
\end{supertabular}


\paragraph{Limite surface taxée}
\noindent
\vspace{\baselineskip}

\renewcommand{\arraystretch}{1.2}
\begin{supertabular}{|p{10cm}|c|p{15cm}|}
\cline{2-3} \multirow{2}{10cm}{Limite surface taxée} & 09\_01\_05\_01 & Limite surface taxée\\


\cline{2-3}                    & 09\_01\_05\_02 & Terrasse en bois\\
\hline
\end{supertabular}


\paragraph{Devanture}
\noindent
\vspace{\baselineskip}

\renewcommand{\arraystretch}{1.2}
\begin{supertabular}{|p{10cm}|c|p{15cm}|}
\cline{2-3} \multirow{3}{10cm}{Devanture} & 09\_01\_06\_01 & Devanture\\


\cline{2-3}                    & 09\_01\_06\_02 & Marquise magasin\\


\cline{2-3}                    & 09\_01\_06\_03 & Tente de marché ou magasin\\
\hline
\end{supertabular}


\paragraph{Ruine}
\noindent
\vspace{\baselineskip}

\renewcommand{\arraystretch}{1.2}
\begin{supertabular}{|p{10cm}|c|p{15cm}|}
\cline{2-3} Ruine & 09\_01\_07\_01 & Ruine\\
\hline
\end{supertabular}


\paragraph{Mur cyclopéen}
\noindent
\vspace{\baselineskip}

\renewcommand{\arraystretch}{1.2}
\begin{supertabular}{|p{10cm}|c|p{15cm}|}
\cline{2-3} \multirow{2}{10cm}{Mur cyclopéen} & 09\_01\_08\_01 & Mur cyclopéen\\


\cline{2-3}                    & 09\_01\_08\_02 & Enrochement\\
\hline
\end{supertabular}


\paragraph{Tombe}
\noindent
\vspace{\baselineskip}

\renewcommand{\arraystretch}{1.2}
\begin{supertabular}{|p{10cm}|c|p{15cm}|}
\cline{2-3} Tombe & 09\_01\_09\_01 & Tombe\\
\hline
\end{supertabular}

\subsubsection{\large Cloture / Element de séparation (objet linéaire)}
\paragraph{Limite de propriété (façade<->domaine public)}
\noindent
\vspace{\baselineskip}

\renewcommand{\arraystretch}{1.2}
\begin{supertabular}{|p{10cm}|c|p{15cm}|}
\cline{2-3} Limite de propriété (façade<->domaine public) & 09\_02\_01\_01 & Limite de propriété (façade<->domaine public)\\
\hline
\end{supertabular}


\paragraph{Limite séparative entre deux propriétés}
\noindent
\vspace{\baselineskip}

\renewcommand{\arraystretch}{1.2}
\begin{supertabular}{|p{10cm}|c|p{15cm}|}
\cline{2-3} Limite séparative entre deux propriétés & 09\_02\_02\_01 & Limite séparative entre deux propriétés\\
\hline
\end{supertabular}


\paragraph{Limite de bâtiment / façade}
\noindent
\vspace{\baselineskip}

\renewcommand{\arraystretch}{1.2}
\begin{supertabular}{|p{10cm}|c|p{15cm}|}
\cline{2-3} Limite de bâtiment / façade & 09\_02\_03\_01 & Limite de bâtiment / façade\\
\hline
\end{supertabular}


\paragraph{Proéminence d un bâtiment par rapport à sa façade levée}
\noindent
\vspace{\baselineskip}

\renewcommand{\arraystretch}{1.2}
\begin{supertabular}{|p{10cm}|c|p{15cm}|}
\cline{2-3} \multirow{2}{10cm}{Proéminence d un bâtiment par rapport à sa façade levée} & 09\_02\_04\_01 & Proéminence d un bâtiment par rapport à sa façade levée\\


\cline{2-3}                    & 09\_02\_04\_02 & Surplomb\\
\hline
\end{supertabular}


\paragraph{Mur}
\noindent
\vspace{\baselineskip}

\renewcommand{\arraystretch}{1.2}
\begin{supertabular}{|p{10cm}|c|p{15cm}|}
\cline{2-3} \multirow{3}{10cm}{Mur} & 09\_02\_05\_01 & Mur\\


\cline{2-3}                    & 09\_02\_05\_02 & Mur de pierre sèche\\


\cline{2-3}                    & 09\_02\_05\_03 & Mur bahut\\
\hline
\end{supertabular}


\paragraph{Mur de soutènement}
\noindent
\vspace{\baselineskip}

\renewcommand{\arraystretch}{1.2}
\begin{supertabular}{|p{10cm}|c|p{15cm}|}
\cline{2-3} \multirow{3}{10cm}{Mur de soutènement} & 09\_02\_06\_01 & Mur de soutènement\\


\cline{2-3}                    & 09\_02\_06\_02 & Digue fleuve\\


\cline{2-3}                    & 09\_02\_06\_03 & Objet linéaire d un pont\\
\hline
\end{supertabular}


\paragraph{Limite épaisseur mur (deuxième trait)}
\noindent
\vspace{\baselineskip}

\renewcommand{\arraystretch}{1.2}
\begin{supertabular}{|p{10cm}|c|p{15cm}|}
\cline{2-3} \multirow{2}{10cm}{Limite épaisseur mur (deuxième trait)} & 09\_02\_07\_01 & Limite épaisseur mur (deuxième trait)\\


\cline{2-3}                    & 09\_02\_07\_02 & Détail intérieur\\
\hline
\end{supertabular}


\paragraph{Clôture}
\noindent
\vspace{\baselineskip}

\renewcommand{\arraystretch}{1.2}
\begin{supertabular}{|p{10cm}|c|p{15cm}|}
\cline{2-3} Clôture & 09\_02\_08\_01 & Clôture\\
\hline
\end{supertabular}


\paragraph{Préau}
\noindent
\vspace{\baselineskip}

\renewcommand{\arraystretch}{1.2}
\begin{supertabular}{|p{10cm}|c|p{15cm}|}
\cline{2-3} \multirow{2}{10cm}{Préau} & 09\_02\_09\_01 & Préau\\


\cline{2-3}                    & 09\_02\_09\_02 & Verrière au sol\\
\hline
\end{supertabular}

\subsubsection{\large Seuil, Entrée}
\paragraph{Bâti Seuil, Entrée}
\noindent
\vspace{\baselineskip}

\renewcommand{\arraystretch}{1.2}
\begin{supertabular}{|p{10cm}|c|p{15cm}|}
\cline{2-3} \multirow{7}{10cm}{Bâti Seuil, Entrée} & 09\_03\_01\_01 & Sortie de garage ou cour\\


\cline{2-3}                    & 09\_03\_01\_02 & Seuil\\


\cline{2-3}                    & 09\_03\_01\_03 & Pergolas\\


\cline{2-3}                    & 09\_03\_01\_04 & Portillon\\


\cline{2-3}                    & 09\_03\_01\_05 & Marches extérieures, escalier\\


\cline{2-3}                    & 09\_03\_01\_06 & Grille de cave\\


\cline{2-3}                    & 09\_03\_01\_07 & Grille/Bouche de ventilation\\
\hline
\end{supertabular}
\subsection{Réseau de Transport Eau et Energie}
\subsubsection{\large Eau potable}
\paragraph{Conduite de transfert (feeder)}
\noindent
\vspace{\baselineskip}

\renewcommand{\arraystretch}{1.2}
\begin{supertabular}{|p{10cm}|c|p{15cm}|}
\cline{2-3} Conduite de transfert (feeder) & 11\_01\_01\_01 & Conduite de transfert (feeder)\\
\hline
\end{supertabular}


\paragraph{Ouvrage d adduction (notion d eau brute, d aqueduc)}
\noindent
\vspace{\baselineskip}

\renewcommand{\arraystretch}{1.2}
\begin{supertabular}{|p{10cm}|c|p{15cm}|}
\cline{2-3} Ouvrage d adduction (notion d eau brute, d aqueduc) & 11\_01\_02\_01 & Ouvrage d adduction (notion d eau brute, d aqueduc)\\
\hline
\end{supertabular}

\subsubsection{\large Gaz}
\paragraph{GDF haute pression transport}
\noindent
\vspace{\baselineskip}

\renewcommand{\arraystretch}{1.2}
\begin{supertabular}{|p{10cm}|c|p{15cm}|}
\cline{2-3} GDF haute pression transport & 11\_02\_01\_01 & GDF haute pression transport\\
\hline
\end{supertabular}


\paragraph{GDF moyenne pression transport}
\noindent
\vspace{\baselineskip}

\renewcommand{\arraystretch}{1.2}
\begin{supertabular}{|p{10cm}|c|p{15cm}|}
\cline{2-3} GDF moyenne pression transport & 11\_02\_02\_01 & GDF moyenne pression transport\\
\hline
\end{supertabular}


\paragraph{Société X}
\noindent
\vspace{\baselineskip}

\renewcommand{\arraystretch}{1.2}
\begin{supertabular}{|p{10cm}|c|p{15cm}|}
\cline{2-3} Société X & 11\_02\_03\_01 & Société X\\
\hline
\end{supertabular}


\paragraph{Société Y}
\noindent
\vspace{\baselineskip}

\renewcommand{\arraystretch}{1.2}
\begin{supertabular}{|p{10cm}|c|p{15cm}|}
\cline{2-3} Société Y & 11\_02\_04\_01 & Société Y\\
\hline
\end{supertabular}

\subsubsection{\large Electricité (RTE = Réseau de Transport d électricité)}
\paragraph{HTB (U > 75 kV)}
\noindent
\vspace{\baselineskip}

\renewcommand{\arraystretch}{1.2}
\begin{supertabular}{|p{10cm}|c|p{15cm}|}
\cline{2-3} \multirow{4}{10cm}{HTB (U > 75 kV)} & 11\_03\_01\_01 & 400 kV (THT = Très Haute Tension)\\


\cline{2-3}                    & 11\_03\_01\_02 & 225 kV (THT = Très Haute Tension)\\


\cline{2-3}                    & 11\_03\_01\_03 & 150 kV (THT = Très Haute Tension)\\


\cline{2-3}                    & 11\_03\_01\_04 & 90 kV (HT = Haute Tension)\\
\hline
\end{supertabular}


\paragraph{HTA (1500 < U < 75 kV)}
\noindent
\vspace{\baselineskip}

\renewcommand{\arraystretch}{1.2}
\begin{supertabular}{|p{10cm}|c|p{15cm}|}
\cline{2-3} \multirow{3}{10cm}{HTA (1500 < U < 75 kV)} & 11\_03\_02\_05 & 63 kV (Haute Tension)\\


\cline{2-3}                    & 11\_03\_02\_06 & Pylône\\


\cline{2-3}                    & 11\_03\_02\_07 & Pylône de caténaire\\
\hline
\end{supertabular}
\subsection{Reseau de Distribution}
\subsubsection{\large Eau potable}
\paragraph{Objet surfacique du réseau}
\noindent
\vspace{\baselineskip}

\renewcommand{\arraystretch}{1.2}
\begin{supertabular}{|p{10cm}|c|p{15cm}|}
\cline{2-3} \multirow{2}{10cm}{Objet surfacique du réseau} & 13\_01\_01\_01 & Station de surpression\\


\cline{2-3}                    & 13\_01\_01\_02 & Station de pompage\\
\hline
\end{supertabular}


\paragraph{Objet linéaire du réseau}
\noindent
\vspace{\baselineskip}

\renewcommand{\arraystretch}{1.2}
\begin{supertabular}{|p{10cm}|c|p{15cm}|}
\cline{2-3} Objet linéaire du réseau & 13\_01\_02\_02 & Branchement\\
\hline
\end{supertabular}


\paragraph{Objet ponctuel du réseau}
\noindent
\vspace{\baselineskip}

\renewcommand{\arraystretch}{1.2}
\begin{supertabular}{|p{10cm}|c|p{15cm}|}
\cline{2-3} \multirow{8}{10cm}{Objet ponctuel du réseau} & 13\_01\_03\_01 & Bouche à clef hexagonale\\


\cline{2-3}                    & 13\_01\_03\_02 & Bouche à clef ronde\\


\cline{2-3}                    & 13\_01\_03\_05 & Bouche incendie / Borne de puisage\\


\cline{2-3}                    & 13\_01\_03\_06 & Compteur réseau\\


\cline{2-3}                    & 13\_01\_03\_11 & Compteur abonné eau potable\\


\cline{2-3}                    & 13\_01\_03\_12 & Coffret\\


\cline{2-3}                    & 13\_01\_03\_13 & Plaque\\


\cline{2-3}                    & 13\_01\_03\_14 & Fontaine\\
\hline
\end{supertabular}

\subsubsection{\large Gaz}
\paragraph{GDF basse pression distribution}
\noindent
\vspace{\baselineskip}

\renewcommand{\arraystretch}{1.2}
\begin{supertabular}{|p{10cm}|c|p{15cm}|}
\cline{2-3} GDF basse pression distribution & 13\_02\_01\_01 & GDF basse pression distribution\\
\hline
\end{supertabular}


\paragraph{Vanne Gaz}
\noindent
\vspace{\baselineskip}

\renewcommand{\arraystretch}{1.2}
\begin{supertabular}{|p{10cm}|c|p{15cm}|}
\cline{2-3} Vanne Gaz & 13\_02\_02\_01 & Vanne Gaz\\
\hline
\end{supertabular}


\paragraph{Branchement individuel GDF}
\noindent
\vspace{\baselineskip}

\renewcommand{\arraystretch}{1.2}
\begin{supertabular}{|p{10cm}|c|p{15cm}|}
\cline{2-3} Branchement individuel GDF & 13\_02\_03\_01 & Branchement individuel GDF\\
\hline
\end{supertabular}


\paragraph{Autre élément}
\noindent
\vspace{\baselineskip}

\renewcommand{\arraystretch}{1.2}
\begin{supertabular}{|p{10cm}|c|p{15cm}|}
\cline{2-3} Autre élément & 13\_02\_04\_01 & Citerne\\
\hline
\end{supertabular}


\paragraph{Gaz basse pression distribution Société Y}
\noindent
\vspace{\baselineskip}

\renewcommand{\arraystretch}{1.2}
\begin{supertabular}{|p{10cm}|c|p{15cm}|}
\cline{2-3} Gaz basse pression distribution Société Y & 13\_02\_05\_01 & Gaz basse pression distribution Société Y\\
\hline
\end{supertabular}

\subsubsection{\large Electricité}
\paragraph{Objet surfacique (réseau de distribution Électricité)}
\noindent
\vspace{\baselineskip}

\renewcommand{\arraystretch}{1.2}
\begin{supertabular}{|p{10cm}|c|p{15cm}|}
\cline{2-3} Objet surfacique (réseau de distribution Électricité) & 13\_03\_01\_01 & Objet surfacique (réseau de distribution Electricité)\\
\hline
\end{supertabular}


\paragraph{Objet linéaire (HTA (1500 < U < 75 kV))}
\noindent
\vspace{\baselineskip}

\renewcommand{\arraystretch}{1.2}
\begin{supertabular}{|p{10cm}|c|p{15cm}|}
\cline{2-3} Objet linéaire (HTA (1500 < U < 75 kV)) & 13\_03\_03\_01 & HTA 20 kV\\
\hline
\end{supertabular}


\paragraph{Objet linéaire (BTB (750 < U < 1500 V))}
\noindent
\vspace{\baselineskip}

\renewcommand{\arraystretch}{1.2}
\begin{supertabular}{|p{10cm}|c|p{15cm}|}
\cline{2-3} Objet linéaire (BTB (750 < U < 1500 V)) & 13\_03\_04\_01 & Objet linéaire (BTB (750 < U < 1500 V))\\
\hline
\end{supertabular}


\paragraph{Objet linéaire (BTA (120 < U < 750 V))}
\noindent
\vspace{\baselineskip}

\renewcommand{\arraystretch}{1.2}
\begin{supertabular}{|p{10cm}|c|p{15cm}|}
\cline{2-3} \multirow{2}{10cm}{Objet linéaire (BTA (120 < U < 750 V))} & 13\_03\_05\_01 & BTA 400 V\\


\cline{2-3}                    & 13\_03\_05\_02 & BTA 230 V\\
\hline
\end{supertabular}


\paragraph{Objet ponctuel}
\noindent
\vspace{\baselineskip}

\renewcommand{\arraystretch}{1.2}
\begin{supertabular}{|p{10cm}|c|p{15cm}|}
\cline{2-3} \multirow{12}{10cm}{Objet ponctuel} & 13\_03\_15\_01 & Borne d alimentation - Prise électrique\\


\cline{2-3}                    & 13\_03\_15\_02 & Borne de distribution Liselec\\


\cline{2-3}                    & 13\_03\_15\_03 & Borne de distribution (marché, camping-car)\\


\cline{2-3}                    & 13\_03\_15\_04 & Armoire de commande EDF\\


\cline{2-3}                    & 13\_03\_15\_05 & Applique (Accrochage du réseau EDF à un point de façade)\\


\cline{2-3}                    & 13\_03\_15\_06 & Poteau EDF\\


\cline{2-3}                    & 13\_03\_15\_07 & Coffret / Compteur abonné / Branchement individuel\\


\cline{2-3}                    & 13\_03\_15\_08 & Regard technique EDF\\


\cline{2-3}                    & 13\_03\_15\_09 & Plaque simple\\


\cline{2-3}                    & 13\_03\_15\_10 & Plaque double\\


\cline{2-3}                    & 13\_03\_15\_11 & Plaque triple\\


\cline{2-3}                    & 13\_03\_15\_12 & Plaque quadruple\\
\hline
\end{supertabular}

\subsubsection{\large Eclairage public}
\paragraph{Objet surfacique (Éclairage public)}
\noindent
\vspace{\baselineskip}

\renewcommand{\arraystretch}{1.2}
\begin{supertabular}{|p{10cm}|c|p{15cm}|}
\cline{2-3} Objet surfacique (Éclairage public) & 13\_04\_01\_01 & Éclairage public\\
\hline
\end{supertabular}


\paragraph{Objet Linéaire}
\noindent
\vspace{\baselineskip}

\renewcommand{\arraystretch}{1.2}
\begin{supertabular}{|p{10cm}|c|p{15cm}|}
\cline{2-3} \multirow{2}{10cm}{Objet Linéaire} & 13\_04\_02\_01 & Cable aérien ou souterrain\\


\cline{2-3}                    & 13\_04\_02\_06 & Branchement\\
\hline
\end{supertabular}


\paragraph{Objet ponctuel}
\noindent
\vspace{\baselineskip}

\renewcommand{\arraystretch}{1.2}
\begin{supertabular}{|p{10cm}|c|p{15cm}|}
\cline{2-3} Objet ponctuel & 13\_04\_03\_01 & Armoire\\
\hline
\end{supertabular}
\subsection{Reseau de Transport d information}
\subsubsection{\large Réseau Ville de La Rochelle}
\paragraph{Poste de Contrôle d Accès (PCA)}
\noindent
\vspace{\baselineskip}

\renewcommand{\arraystretch}{1.2}
\begin{supertabular}{|p{10cm}|c|p{15cm}|}
\cline{2-3} Poste de Contrôle d Accès (PCA) & 15\_01\_01\_01 & Poste de Contrôle d Accès (PCA)\\
\hline
\end{supertabular}


\paragraph{Regard comptage routier}
\noindent
\vspace{\baselineskip}

\renewcommand{\arraystretch}{1.2}
\begin{supertabular}{|p{10cm}|c|p{15cm}|}
\cline{2-3} Regard comptage routier & 15\_01\_02\_01 & Regard comptage routier\\
\hline
\end{supertabular}


\paragraph{Réseau DSI}
\noindent
\vspace{\baselineskip}

\renewcommand{\arraystretch}{1.2}
\begin{supertabular}{|p{10cm}|c|p{15cm}|}
\cline{2-3} Réseau DSI & 15\_01\_03\_01 & Reseau DSI\\
\hline
\end{supertabular}

\subsubsection{\large Réseau opérateur}
\paragraph{Objet surfacique}
\noindent
\vspace{\baselineskip}

\renewcommand{\arraystretch}{1.2}
\begin{supertabular}{|p{10cm}|c|p{15cm}|}
\cline{2-3} Objet surfacique & 15\_02\_01\_01 & Objet surfacique\\
\hline
\end{supertabular}


\paragraph{Objet ponctuel}
\noindent
\vspace{\baselineskip}

\renewcommand{\arraystretch}{1.2}
\begin{supertabular}{|p{10cm}|c|p{15cm}|}
\cline{2-3} \multirow{8}{10cm}{Objet ponctuel} & 15\_02\_03\_01 & Armoire FT\\


\cline{2-3}                    & 15\_02\_03\_03 & Support : poteau EDF\\


\cline{2-3}                    & 15\_02\_03\_04 & Support : potence\\


\cline{2-3}                    & 15\_02\_03\_05 & Support : poteau FT\\


\cline{2-3}                    & 15\_02\_03\_07 & Plaque simple\\


\cline{2-3}                    & 15\_02\_03\_08 & Plaque double\\


\cline{2-3}                    & 15\_02\_03\_09 & Plaque triple\\


\cline{2-3}                    & 15\_02\_03\_10 & Plaque quadruple\\
\hline
\end{supertabular}
\subsection{Reseau en attente}
\subsubsection{\large Fourreau en attente - objet linéaire}
\paragraph{Fourreau en attente - objet linéaire}
\noindent
\vspace{\baselineskip}

\renewcommand{\arraystretch}{1.2}
\begin{supertabular}{|p{10cm}|c|p{15cm}|}
\cline{2-3} Fourreau en attente - objet linéaire & 17\_01\_01\_01 & Fourreau en attente\\
\hline
\end{supertabular}

\subsubsection{\large Fourreau en attente - objet ponctuel}
\paragraph{Fourreau en attente - objet ponctuel}
\noindent
\vspace{\baselineskip}

\renewcommand{\arraystretch}{1.2}
\begin{supertabular}{|p{10cm}|c|p{15cm}|}
\cline{2-3} \multirow{9}{10cm}{Fourreau en attente - objet ponctuel} & 17\_02\_01\_01 & Armoire\\


\cline{2-3}                    & 17\_02\_01\_02 & Coffret\\


\cline{2-3}                    & 17\_02\_01\_03 & Regard rond\\


\cline{2-3}                    & 17\_02\_01\_04 & Regard carré\\


\cline{2-3}                    & 17\_02\_01\_05 & Plaque ronde\\


\cline{2-3}                    & 17\_02\_01\_06 & Plaque simple\\


\cline{2-3}                    & 17\_02\_01\_07 & Plaque double\\


\cline{2-3}                    & 17\_02\_01\_08 & Plaque triple\\


\cline{2-3}                    & 17\_02\_01\_09 & Plaque quadruple\\
\hline
\end{supertabular}
\subsection{Réseau de Collecte}
\subsubsection{\large Réseau d assainissement pluvial}
\paragraph{Objet Linéaire}
\noindent
\vspace{\baselineskip}

\renewcommand{\arraystretch}{1.2}
\begin{supertabular}{|p{10cm}|c|p{15cm}|}
\cline{2-3} \multirow{2}{10cm}{Objet Linéaire} & 19\_01\_02\_03 & Buse\\


\cline{2-3}                    & 19\_01\_02\_04 & Descente d eau\\
\hline
\end{supertabular}


\paragraph{Objet ponctuel}
\noindent
\vspace{\baselineskip}

\renewcommand{\arraystretch}{1.2}
\begin{supertabular}{|p{10cm}|c|p{15cm}|}
\cline{2-3} \multirow{4}{10cm}{Objet ponctuel} & 19\_01\_03\_02 & Plaque\\


\cline{2-3}                    & 19\_01\_03\_04 & Avaloir\\


\cline{2-3}                    & 19\_01\_03\_05 & Grille (ronde, carrée ou rectangulaire)\\


\cline{2-3}                    & 19\_01\_03\_10 & Descente de dalle\\
\hline
\end{supertabular}

\subsubsection{\large Réseau d assainissement eaux usées}
\paragraph{Objet surfacique}
\noindent
\vspace{\baselineskip}

\renewcommand{\arraystretch}{1.2}
\begin{supertabular}{|p{10cm}|c|p{15cm}|}
\cline{2-3} Objet surfacique & 19\_02\_01\_01 & Objet surfacique\\
\hline
\end{supertabular}


\paragraph{Objet Linéaire}
\noindent
\vspace{\baselineskip}

\renewcommand{\arraystretch}{1.2}
\begin{supertabular}{|p{10cm}|c|p{15cm}|}
\cline{2-3} Objet Linéaire & 19\_02\_02\_01 & Collecteur Eaux Usées\\
\hline
\end{supertabular}


\paragraph{Objet ponctuel}
\noindent
\vspace{\baselineskip}

\renewcommand{\arraystretch}{1.2}
\begin{supertabular}{|p{10cm}|c|p{15cm}|}
\cline{2-3} Objet ponctuel & 19\_02\_03\_01 & Regard de Visite Eaux Usées\\
\hline
\end{supertabular}
\subsection{Signalisation lumineuse}
\subsubsection{\large Information parking - Panneau à messages variables (PMV)}
\paragraph{Panneau à messages variables (PMV)}
\noindent
\vspace{\baselineskip}

\renewcommand{\arraystretch}{1.2}
\begin{supertabular}{|p{10cm}|c|p{15cm}|}
\cline{2-3} Panneau à messages variables (PMV) & 21\_01\_01\_01 & Panneau à messages variables (PMV)\\
\hline
\end{supertabular}

\subsubsection{\large Feu tricolore - Feu et signal lumineux (R11 à R24)}
\paragraph{Armoire de commande Signalisation Tricolore}
\noindent
\vspace{\baselineskip}

\renewcommand{\arraystretch}{1.2}
\begin{supertabular}{|p{10cm}|c|p{15cm}|}
\cline{2-3} Armoire de commande Signalisation Tricolore & 21\_02\_01\_01 & Armoire de commande Signalisation Tricolore\\
\hline
\end{supertabular}


\paragraph{Regard Technique Signalisation Tricolore}
\noindent
\vspace{\baselineskip}

\renewcommand{\arraystretch}{1.2}
\begin{supertabular}{|p{10cm}|c|p{15cm}|}
\cline{2-3} Regard Technique Signalisation Tricolore & 21\_02\_02\_01 & Regard Technique Signalisation Tricolore\\
\hline
\end{supertabular}


\paragraph{Feu et signal lumineux (R11 à R24) accroché à un point de référence (support ou applique)}
\noindent
\vspace{\baselineskip}

\renewcommand{\arraystretch}{1.2}
\begin{supertabular}{|p{10cm}|c|p{15cm}|}
\cline{2-3} Feu et signal lumineux (R11 à R24) accroché à un point de référence (support ou applique) & 21\_02\_03\_01 & Feu de signalisation\\
\hline
\end{supertabular}
\subsection{Signalisation verticale}
\subsubsection{\large Signalisation verticale de police}
\paragraph{Panneaux de danger (A1a à A24)}
\noindent
\vspace{\baselineskip}

\renewcommand{\arraystretch}{1.2}
\begin{supertabular}{|p{10cm}|c|p{15cm}|}
\cline{2-3} Panneaux de danger (A1a à A24) & 23\_01\_01\_01 & Panneaux de danger (A1a à A24)\\
\hline
\end{supertabular}


\paragraph{Panneaux d intersection et de priorité (AB1 à AB25)}
\noindent
\vspace{\baselineskip}

\renewcommand{\arraystretch}{1.2}
\begin{supertabular}{|p{10cm}|c|p{15cm}|}
\cline{2-3} Panneaux d intersection et de priorité (AB1 à AB25) & 23\_01\_02\_01 & Panneaux d intersection et de priorité (AB1 à AB25)\\
\hline
\end{supertabular}


\paragraph{Panneaux de prescription (B0 à B51)}
\noindent
\vspace{\baselineskip}

\renewcommand{\arraystretch}{1.2}
\begin{supertabular}{|p{10cm}|c|p{15cm}|}
\cline{2-3} Panneaux de prescription (B0 à B51) & 23\_01\_03\_01 & Panneaux de prescription (B0 à B51)\\
\hline
\end{supertabular}


\paragraph{Panneaux d indication (C1a à CE50)}
\noindent
\vspace{\baselineskip}

\renewcommand{\arraystretch}{1.2}
\begin{supertabular}{|p{10cm}|c|p{15cm}|}
\cline{2-3} Panneaux d indication (C1a à CE50) & 23\_01\_04\_01 & Panneaux d indication (C1a à CE50)\\
\hline
\end{supertabular}


\paragraph{Panonceaux (M1 à M11b)}
\noindent
\vspace{\baselineskip}

\renewcommand{\arraystretch}{1.2}
\begin{supertabular}{|p{10cm}|c|p{15cm}|}
\cline{2-3} Panonceaux (M1 à M11b) & 23\_01\_05\_01 & Panonceaux (M1 à M11b)\\
\hline
\end{supertabular}


\paragraph{Symboles (SE2b à SU1)}
\noindent
\vspace{\baselineskip}

\renewcommand{\arraystretch}{1.2}
\begin{supertabular}{|p{10cm}|c|p{15cm}|}
\cline{2-3} Symboles (SE2b à SU1) & 23\_01\_06\_01 & Symboles (SE2b à SU1)\\
\hline
\end{supertabular}


\paragraph{Autres objets}
\noindent
\vspace{\baselineskip}

\renewcommand{\arraystretch}{1.2}
\begin{supertabular}{|p{10cm}|c|p{15cm}|}
\cline{2-3} Autres objets & 23\_01\_07\_01 & Emprise arrêté\\
\hline
\end{supertabular}

\subsubsection{\large Signalisation verticale de direction}
\paragraph{Panneaux de direction (D21a à D79b)}
\noindent
\vspace{\baselineskip}

\renewcommand{\arraystretch}{1.2}
\begin{supertabular}{|p{10cm}|c|p{15cm}|}
\cline{2-3} Panneaux de direction (D21a à D79b) & 23\_02\_01\_01 & Panneaux de direction (D21a à D79b)\\
\hline
\end{supertabular}


\paragraph{Panneaux de jalonnement piétonnier (Dp1a à Dp2b)}
\noindent
\vspace{\baselineskip}

\renewcommand{\arraystretch}{1.2}
\begin{supertabular}{|p{10cm}|c|p{15cm}|}
\cline{2-3} Panneaux de jalonnement piétonnier (Dp1a à Dp2b) & 23\_02\_02\_01 & Panneaux de jalonnement piétonnier (Dp1a à Dp2b)\\
\hline
\end{supertabular}


\paragraph{Panneaux de jalonnement pour cyclistes (Dv11 à Dv61)}
\noindent
\vspace{\baselineskip}

\renewcommand{\arraystretch}{1.2}
\begin{supertabular}{|p{10cm}|c|p{15cm}|}
\cline{2-3} Panneaux de jalonnement pour cyclistes (Dv11 à Dv61) & 23\_02\_03\_01 & Panneaux de jalonnement pour cyclistes (Dv11 à Dv61)\\
\hline
\end{supertabular}


\paragraph{Panneaux de localisation (E31 à EB20)}
\noindent
\vspace{\baselineskip}

\renewcommand{\arraystretch}{1.2}
\begin{supertabular}{|p{10cm}|c|p{15cm}|}
\cline{2-3} Panneaux de localisation (E31 à EB20) & 23\_02\_04\_01 & Panneaux de localisation (E31 à EB20)\\
\hline
\end{supertabular}


\paragraph{Panneaux d information (H11 à H33)}
\noindent
\vspace{\baselineskip}

\renewcommand{\arraystretch}{1.2}
\begin{supertabular}{|p{10cm}|c|p{15cm}|}
\cline{2-3} Panneaux d information (H11 à H33) & 23\_02\_05\_01 & Panneaux d information (H11 à H33)\\
\hline
\end{supertabular}


\paragraph{Idéogrammes, emblèmes ou logotypes (ID1 à ID15a41 + Logotype)}
\noindent
\vspace{\baselineskip}

\renewcommand{\arraystretch}{1.2}
\begin{supertabular}{|p{10cm}|c|p{15cm}|}
\cline{2-3} Idéogrammes, emblèmes ou logotypes (ID1 à ID15a41 + Logotype) & 23\_02\_06\_01 & Idéogrammes, emblèmes ou logotypes (ID1 à ID15a41 + Logotype)\\
\hline
\end{supertabular}


\paragraph{Autres objets}
\noindent
\vspace{\baselineskip}

\renewcommand{\arraystretch}{1.2}
\begin{supertabular}{|p{10cm}|c|p{15cm}|}
\cline{2-3} Autres objets & 23\_02\_07\_01 & Autres objets\\
\hline
\end{supertabular}

\subsubsection{\large Autres types de signal routier}
\paragraph{Balises (J1 à J14b)}
\noindent
\vspace{\baselineskip}

\renewcommand{\arraystretch}{1.2}
\begin{supertabular}{|p{10cm}|c|p{15cm}|}
\cline{2-3} \multirow{2}{10cm}{Balises (J1 à J14b)} & 23\_03\_01\_01 & Balise\\


\cline{2-3}                    & 23\_03\_01\_02 & Chicanes – balisettes\\
\hline
\end{supertabular}


\paragraph{Feux de balisage et d alerte (G1 à R2d)}
\noindent
\vspace{\baselineskip}

\renewcommand{\arraystretch}{1.2}
\begin{supertabular}{|p{10cm}|c|p{15cm}|}
\cline{2-3} Feux de balisage et d alerte (G1 à R2d) & 23\_03\_02\_01 & Feux de balisage et d alerte (G1 à R2d)\\
\hline
\end{supertabular}


\paragraph{Panneaux à messages variables (PMV)}
\noindent
\vspace{\baselineskip}

\renewcommand{\arraystretch}{1.2}
\begin{supertabular}{|p{10cm}|c|p{15cm}|}
\cline{2-3} Panneaux à messages variables (PMV) & 23\_03\_03\_01 & Panneaux à messages variables (PMV)\\
\hline
\end{supertabular}


\paragraph{Panneaux et dispositifs de signalisation temporaire (AK2 à KR11)}
\noindent
\vspace{\baselineskip}

\renewcommand{\arraystretch}{1.2}
\begin{supertabular}{|p{10cm}|c|p{15cm}|}
\cline{2-3} Panneaux et dispositifs de signalisation temporaire (AK2 à KR11) & 23\_03\_04\_01 & Panneaux et dispositifs de signalisation temporaire (AK2 à KR11)\\
\hline
\end{supertabular}
\subsection{Signalisation horizontale}
\subsubsection{\large Lignes longitudinales axiales}
\paragraph{Lignes continues (continue)}
\noindent
\vspace{\baselineskip}

\renewcommand{\arraystretch}{1.2}
\begin{supertabular}{|p{10cm}|c|p{15cm}|}
\cline{2-3} \multirow{2}{10cm}{Lignes continues (continue)} & 25\_01\_01\_01 & Ligne axiale ou de délimitation des voies\\


\cline{2-3}                    & 25\_01\_01\_02 & Ligne axiale sur chaussée à 4 voies\\
\hline
\end{supertabular}


\paragraph{Lignes discontinues de type T1 (T1, Tp1, ou T3)}
\noindent
\vspace{\baselineskip}

\renewcommand{\arraystretch}{1.2}
\begin{supertabular}{|p{10cm}|c|p{15cm}|}
\cline{2-3} \multirow{2}{10cm}{Lignes discontinues de type T1 (T1, Tp1, ou T3)} & 25\_01\_02\_01 & Ligne axiales ou de délimitation de voie en agglomération\\


\cline{2-3}                    & 25\_01\_02\_02 & Ligne axiales ou de délimitation de voie en rase campagne\\
\hline
\end{supertabular}


\paragraph{Lignes discontinues de type T3 (T3)}
\noindent
\vspace{\baselineskip}

\renewcommand{\arraystretch}{1.2}
\begin{supertabular}{|p{10cm}|c|p{15cm}|}
\cline{2-3} \multirow{2}{10cm}{Lignes discontinues de type T3 (T3)} & 25\_01\_03\_01 & Ligne d annonce d une ligne continue\\


\cline{2-3}                    & 25\_01\_03\_02 & Ligne de dissuasion en remplacement d une ligne continue\\
\hline
\end{supertabular}


\paragraph{Lignes mixtes (T1 ou T3)}
\noindent
\vspace{\baselineskip}

\renewcommand{\arraystretch}{1.2}
\begin{supertabular}{|p{10cm}|c|p{15cm}|}
\cline{2-3} Lignes mixtes (T1 ou T3) & 25\_01\_04\_01 & La ligne mixte est constituée par une ligne continue doublée par une ligne discontinue de type T1 ou T3\\
\hline
\end{supertabular}

\subsubsection{\large Lignes continues délimitant le Terre Plein Central, les ilots ou certains couloirs réservés}
\paragraph{Ligne de délimitation de terre-plein central (continue)}
\noindent
\vspace{\baselineskip}

\renewcommand{\arraystretch}{1.2}
\begin{supertabular}{|p{10cm}|c|p{15cm}|}
\cline{2-3} Ligne de délimitation de terre-plein central (continue) & 25\_02\_01\_01 & Ligne de délimitation de terre-plein central (continue)\\
\hline
\end{supertabular}


\paragraph{Ligne de délimitation du contour des îlots (continue)}
\noindent
\vspace{\baselineskip}

\renewcommand{\arraystretch}{1.2}
\begin{supertabular}{|p{10cm}|c|p{15cm}|}
\cline{2-3} Ligne de délimitation du contour des îlots (continue) & 25\_02\_02\_01 & Ligne de délimitation du contour des îlots (continue)\\
\hline
\end{supertabular}


\paragraph{Ligne de délimitation de certains couloirs réservés (continue)}
\noindent
\vspace{\baselineskip}

\renewcommand{\arraystretch}{1.2}
\begin{supertabular}{|p{10cm}|c|p{15cm}|}
\cline{2-3} Ligne de délimitation de certains couloirs réservés (continue) & 25\_02\_03\_01 & Ligne de délimitation de certains couloirs réservés (continue)\\
\hline
\end{supertabular}

\subsubsection{\large Lignes longitudinales de rives ou de délimitation de certaines voies}
\paragraph{Lignes discontinues de type T2 (T2)}
\noindent
\vspace{\baselineskip}

\renewcommand{\arraystretch}{1.2}
\begin{supertabular}{|p{10cm}|c|p{15cm}|}
\cline{2-3} \multirow{3}{10cm}{Lignes discontinues de type T2 (T2)} & 25\_03\_01\_01 & Ligne d entrée et de sortie des voies pour véhicules lents\\


\cline{2-3}                    & 25\_03\_01\_02 & Ligne de délimitation des voies de décélération, d insertion ou d entrecroisement\\


\cline{2-3}                    & 25\_03\_01\_03 & Ligne de rive de chaussée\\
\hline
\end{supertabular}


\paragraph{Lignes discontinues de type T3 (T3 ou Tp3)}
\noindent
\vspace{\baselineskip}

\renewcommand{\arraystretch}{1.2}
\begin{supertabular}{|p{10cm}|c|p{15cm}|}
\cline{2-3} \multirow{5}{10cm}{Lignes discontinues de type T3 (T3 ou Tp3)} & 25\_03\_02\_01 & Ligne de délimitation dans certains cas d un couloir réservé aux autobus\\


\cline{2-3}                    & 25\_03\_02\_02 & Ligne de délimitation dans certains cas de bandes cyclables\\


\cline{2-3}                    & 25\_03\_02\_03 & Ligne de délimitation de voies pour véhicules lents sur lesquelles il n y a pas d interdiction de dépasser\\


\cline{2-3}                    & 25\_03\_02\_04 & Ligne de rive aux approches de certains carrefours\\


\cline{2-3}                    & 25\_03\_02\_05 & Ligne délimitant une bande d arrêt d urgence, ligne de rive sur autoroute\\
\hline
\end{supertabular}

\subsubsection{\large Présignalisation des points singuliers}
\paragraph{Ligne d annonce d une ligne continue (T3)}
\noindent
\vspace{\baselineskip}

\renewcommand{\arraystretch}{1.2}
\begin{supertabular}{|p{10cm}|c|p{15cm}|}
\cline{2-3} Ligne d annonce d une ligne continue (T3) & 25\_04\_01\_01 & Ligne d annonce d une ligne continue (T3)\\
\hline
\end{supertabular}


\paragraph{Flèches de rabattement}
\noindent
\vspace{\baselineskip}

\renewcommand{\arraystretch}{1.2}
\begin{supertabular}{|p{10cm}|c|p{15cm}|}
\cline{2-3} Flèches de rabattement & 25\_04\_02\_01 & Flèches de rabattement\\
\hline
\end{supertabular}


\paragraph{Flèches directionnelles}
\noindent
\vspace{\baselineskip}

\renewcommand{\arraystretch}{1.2}
\begin{supertabular}{|p{10cm}|c|p{15cm}|}
\cline{2-3} Flèches directionnelles & 25\_04\_03\_01 & Flèches directionnelles\\
\hline
\end{supertabular}

\subsubsection{\large Lignes Transversales}
\paragraph{Ligne \_STOP\_ (continue)}
\noindent
\vspace{\baselineskip}

\renewcommand{\arraystretch}{1.2}
\begin{supertabular}{|p{10cm}|c|p{15cm}|}
\cline{2-3} Ligne \_STOP\_ (continue) & 25\_05\_01\_01 & Ligne \_STOP\_ (continue)\\
\hline
\end{supertabular}


\paragraph{Ligne \_Cédez-le-passage\_ (Tp2)}
\noindent
\vspace{\baselineskip}

\renewcommand{\arraystretch}{1.2}
\begin{supertabular}{|p{10cm}|c|p{15cm}|}
\cline{2-3} Ligne \_Cédez-le-passage\_ (Tp2) & 25\_05\_02\_01 & Ligne \_Cédez-le-passage\_ (Tp2)\\
\hline
\end{supertabular}


\paragraph{Ligne d effet des feux (Tp2)}
\noindent
\vspace{\baselineskip}

\renewcommand{\arraystretch}{1.2}
\begin{supertabular}{|p{10cm}|c|p{15cm}|}
\cline{2-3} Ligne d effet des feux (Tp2) & 25\_05\_03\_01 & Ligne d effet des feux (Tp2)\\
\hline
\end{supertabular}


\paragraph{Ligne de guidage en intersection Tourne à gauche à l indonésienne - carrefour en baïonnette (Tp2)}
\noindent
\vspace{\baselineskip}

\renewcommand{\arraystretch}{1.2}
\begin{supertabular}{|p{10cm}|c|p{15cm}|}
\cline{2-3} Ligne de guidage en intersection Tourne à gauche à l indonésienne - carrefour en baïonnette (Tp2) & 25\_05\_04\_01 & Ligne de guidage en intersection Tourne à gauche à l indonésienne - carrefour en baïonnette (Tp2)\\
\hline
\end{supertabular}


\paragraph{Ligne de début et de fin de voie cyclable (Tp2)}
\noindent
\vspace{\baselineskip}

\renewcommand{\arraystretch}{1.2}
\begin{supertabular}{|p{10cm}|c|p{15cm}|}
\cline{2-3} Ligne de début et de fin de voie cyclable (Tp2) & 25\_05\_05\_01 & Ligne de début et de fin de voie cyclable (Tp2)\\
\hline
\end{supertabular}

\subsubsection{\large Passages pour piétons}
\paragraph{Passages pour piétons}
\noindent
\vspace{\baselineskip}

\renewcommand{\arraystretch}{1.2}
\begin{supertabular}{|p{10cm}|c|p{15cm}|}
\cline{2-3} Passages pour piétons & 25\_06\_01\_01 & Passages pour piétons\\
\hline
\end{supertabular}

\subsubsection{\large Marques de ralentisseur de type dos-d âne}
\paragraph{Marques de ralentisseur de type dos-d âne}
\noindent
\vspace{\baselineskip}

\renewcommand{\arraystretch}{1.2}
\begin{supertabular}{|p{10cm}|c|p{15cm}|}
\cline{2-3} Marques de ralentisseur de type dos-d âne & 25\_07\_01\_01 & Marques de ralentisseur de type dos-d âne\\
\hline
\end{supertabular}

\subsubsection{\large Marques relatives aux transports en commun}
\paragraph{Ligne marquant l emplacement d un arrêt d autobus (jaune, continue, zigzag)}
\noindent
\vspace{\baselineskip}

\renewcommand{\arraystretch}{1.2}
\begin{supertabular}{|p{10cm}|c|p{15cm}|}
\cline{2-3} Ligne marquant l emplacement d un arrêt d autobus (jaune, continue, zigzag) & 25\_08\_01\_01 & Ligne marquant l emplacement d un arrêt d autobus (jaune, continue, zigzag)\\
\hline
\end{supertabular}


\paragraph{Franchissement des carrefours par les voies réservées (blanc, damier)}
\noindent
\vspace{\baselineskip}

\renewcommand{\arraystretch}{1.2}
\begin{supertabular}{|p{10cm}|c|p{15cm}|}
\cline{2-3} Franchissement des carrefours par les voies réservées (blanc, damier) & 25\_08\_02\_01 & Franchissement des carrefours par les voies réservées (blanc, damier)\\
\hline
\end{supertabular}


\paragraph{Inscriptions sur les voies réservées (\_BUS\_)}
\noindent
\vspace{\baselineskip}

\renewcommand{\arraystretch}{1.2}
\begin{supertabular}{|p{10cm}|c|p{15cm}|}
\cline{2-3} Inscriptions sur les voies réservées (\_BUS\_) & 25\_08\_03\_01 & Inscriptions sur les voies réservées (\_BUS\_)\\
\hline
\end{supertabular}

\subsubsection{\large Marques relatives au stationnement}
\paragraph{Ligne délimitant les places de stationnement (blanche ou bleue, Tp2 ou continue)}
\noindent
\vspace{\baselineskip}

\renewcommand{\arraystretch}{1.2}
\begin{supertabular}{|p{10cm}|c|p{15cm}|}
\cline{2-3} Ligne délimitant les places de stationnement (blanche ou bleue, Tp2 ou continue) & 25\_09\_01\_01 & Ligne délimitant les places de stationnement (blanche ou bleue, Tp2 ou continue)\\
\hline
\end{supertabular}


\paragraph{Ligne confirmant ou indiquant l interdiction de stationner (jaune, Tp2)}
\noindent
\vspace{\baselineskip}

\renewcommand{\arraystretch}{1.2}
\begin{supertabular}{|p{10cm}|c|p{15cm}|}
\cline{2-3} Ligne confirmant ou indiquant l interdiction de stationner (jaune, Tp2) & 25\_09\_02\_01 & Ligne confirmant ou indiquant l interdiction de stationner (jaune, Tp2)\\
\hline
\end{supertabular}


\paragraph{Ligne confirmant ou indiquant l interdiction de s arrêter (jaune, continue)}
\noindent
\vspace{\baselineskip}

\renewcommand{\arraystretch}{1.2}
\begin{supertabular}{|p{10cm}|c|p{15cm}|}
\cline{2-3} Ligne confirmant ou indiquant l interdiction de s arrêter (jaune, continue) & 25\_09\_03\_01 & Ligne confirmant ou indiquant l interdiction de s arrêter (jaune, continue)\\
\hline
\end{supertabular}

\subsubsection{\large Marques pour voies cyclables}
\paragraph{Traversées de chaussées par les voies cyclables (Tp2)}
\noindent
\vspace{\baselineskip}

\renewcommand{\arraystretch}{1.2}
\begin{supertabular}{|p{10cm}|c|p{15cm}|}
\cline{2-3} Traversées de chaussées par les voies cyclables (Tp2) & 25\_10\_01\_01 & Traversées de chaussées par les voies cyclables (Tp2)\\
\hline
\end{supertabular}
\subsection{Stationnement}
\subsubsection{\large Zone de stationnement (plusieurs emplacements)}
\paragraph{Contour de parking}
\noindent
\vspace{\baselineskip}

\renewcommand{\arraystretch}{1.2}
\begin{supertabular}{|p{10cm}|c|p{15cm}|}
\cline{2-3} Contour de parking & 27\_01\_01\_01 & Zone de stationnement\\
\hline
\end{supertabular}


\paragraph{Zone de livraison}
\noindent
\vspace{\baselineskip}

\renewcommand{\arraystretch}{1.2}
\begin{supertabular}{|p{10cm}|c|p{15cm}|}
\cline{2-3} Zone de livraison & 27\_01\_02\_01 & Zone de livraison\\
\hline
\end{supertabular}


\paragraph{Horodateur}
\noindent
\vspace{\baselineskip}

\renewcommand{\arraystretch}{1.2}
\begin{supertabular}{|p{10cm}|c|p{15cm}|}
\cline{2-3} Horodateur & 27\_01\_03\_01 & Horodateur\\
\hline
\end{supertabular}


\paragraph{Péage}
\noindent
\vspace{\baselineskip}

\renewcommand{\arraystretch}{1.2}
\begin{supertabular}{|p{10cm}|c|p{15cm}|}
\cline{2-3} Péage & 27\_01\_04\_01 & Péage\\
\hline
\end{supertabular}

\subsubsection{\large Emplacement de stationnement}
\paragraph{Emplacement normal}
\noindent
\vspace{\baselineskip}

\renewcommand{\arraystretch}{1.2}
\begin{supertabular}{|p{10cm}|c|p{15cm}|}
\cline{2-3} Emplacement normal & 27\_02\_01\_01 & Emplacement normal\\
\hline
\end{supertabular}


\paragraph{Emplacement PMR}
\noindent
\vspace{\baselineskip}

\renewcommand{\arraystretch}{1.2}
\begin{supertabular}{|p{10cm}|c|p{15cm}|}
\cline{2-3} Emplacement PMR & 27\_02\_02\_01 & Emplacement PMR\\
\hline
\end{supertabular}


\paragraph{Emplacement Taxi}
\noindent
\vspace{\baselineskip}

\renewcommand{\arraystretch}{1.2}
\begin{supertabular}{|p{10cm}|c|p{15cm}|}
\cline{2-3} Emplacement Taxi & 27\_02\_03\_01 & Emplacement Taxi\\
\hline
\end{supertabular}


\paragraph{Emplacement Transporteur de fond}
\noindent
\vspace{\baselineskip}

\renewcommand{\arraystretch}{1.2}
\begin{supertabular}{|p{10cm}|c|p{15cm}|}
\cline{2-3} Emplacement Transporteur de fond & 27\_02\_04\_01 & Emplacement Transporteur de fond\\
\hline
\end{supertabular}


\paragraph{Emplacement Livraison}
\noindent
\vspace{\baselineskip}

\renewcommand{\arraystretch}{1.2}
\begin{supertabular}{|p{10cm}|c|p{15cm}|}
\cline{2-3} Emplacement Livraison & 27\_02\_05\_01 & Emplacement Livraison\\
\hline
\end{supertabular}


\paragraph{Emplacement Véhicule électrique}
\noindent
\vspace{\baselineskip}

\renewcommand{\arraystretch}{1.2}
\begin{supertabular}{|p{10cm}|c|p{15cm}|}
\cline{2-3} Emplacement Véhicule électrique & 27\_02\_06\_01 & Emplacement Véhicule électrique\\
\hline
\end{supertabular}
\subsection{Espaces verts}
\subsubsection{\large Typologie Espaces Verts}
\paragraph{Parcs, jardins et squares}
\noindent
\vspace{\baselineskip}

\renewcommand{\arraystretch}{1.2}
\begin{supertabular}{|p{10cm}|c|p{15cm}|}
\cline{2-3} Parcs, jardins et squares & 29\_01\_01\_01 & Parcs, jardins et squares\\
\hline
\end{supertabular}


\paragraph{Accompagnement de voies}
\noindent
\vspace{\baselineskip}

\renewcommand{\arraystretch}{1.2}
\begin{supertabular}{|p{10cm}|c|p{15cm}|}
\cline{2-3} Accompagnement de voies & 29\_01\_02\_01 & Accompagnement de voies\\
\hline
\end{supertabular}


\paragraph{Accompagnement bâtiments publics}
\noindent
\vspace{\baselineskip}

\renewcommand{\arraystretch}{1.2}
\begin{supertabular}{|p{10cm}|c|p{15cm}|}
\cline{2-3} Accompagnement bâtiments publics & 29\_01\_03\_01 & Accompagnement bâtiments publics\\
\hline
\end{supertabular}


\paragraph{Accompagnement habitations}
\noindent
\vspace{\baselineskip}

\renewcommand{\arraystretch}{1.2}
\begin{supertabular}{|p{10cm}|c|p{15cm}|}
\cline{2-3} Accompagnement habitations & 29\_01\_04\_01 & Accompagnement habitations\\
\hline
\end{supertabular}


\paragraph{Accompagnement établissement industriels et commerciaux}
\noindent
\vspace{\baselineskip}

\renewcommand{\arraystretch}{1.2}
\begin{supertabular}{|p{10cm}|c|p{15cm}|}
\cline{2-3} Accompagnement établissement industriels et commerciaux & 29\_01\_05\_01 & Accompagnement établissement industriels et commerciaux\\
\hline
\end{supertabular}


\paragraph{Espaces verts établissements sociaux ou éducatifs}
\noindent
\vspace{\baselineskip}

\renewcommand{\arraystretch}{1.2}
\begin{supertabular}{|p{10cm}|c|p{15cm}|}
\cline{2-3} Espaces verts établissements sociaux ou éducatifs & 29\_01\_06\_01 & Espaces verts établissements sociaux ou éducatifs\\
\hline
\end{supertabular}


\paragraph{Sports}
\noindent
\vspace{\baselineskip}

\renewcommand{\arraystretch}{1.2}
\begin{supertabular}{|p{10cm}|c|p{15cm}|}
\cline{2-3} Sports & 29\_01\_07\_01 & Sports\\
\hline
\end{supertabular}


\paragraph{Cimetières}
\noindent
\vspace{\baselineskip}

\renewcommand{\arraystretch}{1.2}
\begin{supertabular}{|p{10cm}|c|p{15cm}|}
\cline{2-3} Cimetières & 29\_01\_08\_01 & Cimetières\\
\hline
\end{supertabular}


\paragraph{Campings}
\noindent
\vspace{\baselineskip}

\renewcommand{\arraystretch}{1.2}
\begin{supertabular}{|p{10cm}|c|p{15cm}|}
\cline{2-3} Campings & 29\_01\_09\_01 & Campings\\
\hline
\end{supertabular}


\paragraph{Établissements horticoles}
\noindent
\vspace{\baselineskip}

\renewcommand{\arraystretch}{1.2}
\begin{supertabular}{|p{10cm}|c|p{15cm}|}
\cline{2-3} Établissements horticoles & 29\_01\_10\_01 & Établissements horticoles\\
\hline
\end{supertabular}


\paragraph{Espaces naturels aménagés}
\noindent
\vspace{\baselineskip}

\renewcommand{\arraystretch}{1.2}
\begin{supertabular}{|p{10cm}|c|p{15cm}|}
\cline{2-3} Espaces naturels aménagés & 29\_01\_11\_01 & Espaces naturels aménagés\\
\hline
\end{supertabular}


\paragraph{Jardins familiaux}
\noindent
\vspace{\baselineskip}

\renewcommand{\arraystretch}{1.2}
\begin{supertabular}{|p{10cm}|c|p{15cm}|}
\cline{2-3} Jardins familiaux & 29\_01\_12\_01 & Jardins familiaux\\
\hline
\end{supertabular}


\paragraph{Arbres d alignement}
\noindent
\vspace{\baselineskip}

\renewcommand{\arraystretch}{1.2}
\begin{supertabular}{|p{10cm}|c|p{15cm}|}
\cline{2-3} Arbres d alignement & 29\_01\_13\_01 & Arbres d alignement\\
\hline
\end{supertabular}

\subsubsection{\large Objets topographiques espaces verts (necessitant une précision décimétrique)}
\paragraph{Surfacique}
\noindent
\vspace{\baselineskip}

\renewcommand{\arraystretch}{1.2}
\begin{supertabular}{|p{10cm}|c|p{15cm}|}
\cline{2-3} \multirow{12}{10cm}{Surfacique} & 29\_02\_01\_01 & Locaux / Cabane\\


\cline{2-3}                    & 29\_02\_01\_02 & Équipement de sport\\


\cline{2-3}                    & 29\_02\_01\_03 & Boisement\\


\cline{2-3}                    & 29\_02\_01\_04 & Espaces verts - massifs\\


\cline{2-3}                    & 29\_02\_01\_05 & Lac / Étang / Bassin d orage\\


\cline{2-3}                    & 29\_02\_01\_06 & Bassin / Pataugeoire\\


\cline{2-3}                    & 29\_02\_01\_07 & Zone Arrosage automatique\\


\cline{2-3}                    & 29\_02\_01\_08 & Pelouses\\


\cline{2-3}                    & 29\_02\_01\_09 & Aire de jeu / Bac à sable\\


\cline{2-3}                    & 29\_02\_01\_10 & Sols minéraux (stabilisés ou minéral)\\


\cline{2-3}                    & 29\_02\_01\_11 & Allée\\


\cline{2-3}                    & 29\_02\_01\_12 & Talus\\
\hline
\end{supertabular}


\paragraph{Linéaire}
\noindent
\vspace{\baselineskip}

\renewcommand{\arraystretch}{1.2}
\begin{supertabular}{|p{10cm}|c|p{15cm}|}
\cline{2-3} \multirow{6}{10cm}{Linéaire} & 29\_02\_02\_01 & Haie\\


\cline{2-3}                    & 29\_02\_02\_02 & Tribune stade\\


\cline{2-3}                    & 29\_02\_02\_03 & Fouilles archéologiques\\


\cline{2-3}                    & 29\_02\_02\_04 & Haut de talus\\


\cline{2-3}                    & 29\_02\_02\_05 & Pied de talus\\


\cline{2-3}                    & 29\_02\_02\_06 & Limite de culture\\
\hline
\end{supertabular}


\paragraph{Ponctuel}
\noindent
\vspace{\baselineskip}

\renewcommand{\arraystretch}{1.2}
\begin{supertabular}{|p{10cm}|c|p{15cm}|}
\cline{2-3} \multirow{4}{10cm}{Ponctuel} & 29\_02\_03\_01 & Arbre isolé\\


\cline{2-3}                    & 29\_02\_03\_02 & Bouche d arrosage\\


\cline{2-3}                    & 29\_02\_03\_03 & Fontaine\\


\cline{2-3}                    & 29\_02\_03\_03 & Puit\\
\hline
\end{supertabular}

\subsubsection{\large Espaces verts Objets topographiques espaces verts (necessitant une précision métrique)}
\paragraph{Surfacique}
\noindent
\vspace{\baselineskip}

\renewcommand{\arraystretch}{1.2}
\begin{supertabular}{|p{10cm}|c|p{15cm}|}
\cline{2-3} Surfacique & 29\_03\_01\_01 & Plage, surface de sable\\
\hline
\end{supertabular}


\paragraph{Linéaire}
\noindent
\vspace{\baselineskip}

\renewcommand{\arraystretch}{1.2}
\begin{supertabular}{|p{10cm}|c|p{15cm}|}
\cline{2-3} Linéaire & 29\_03\_02\_01 & Limite Plage, limite eau\\
\hline
\end{supertabular}


\paragraph{Ponctuel}
\noindent
\vspace{\baselineskip}

\renewcommand{\arraystretch}{1.2}
\begin{supertabular}{|p{10cm}|c|p{15cm}|}
\cline{2-3} Ponctuel & 29\_03\_03\_01 & Bouée\\
\hline
\end{supertabular}

\subsubsection{\large Gestion}
\paragraph{Espaces Verts entretien Ville La Rochelle}
\noindent
\vspace{\baselineskip}

\renewcommand{\arraystretch}{1.2}
\begin{supertabular}{|p{10cm}|c|p{15cm}|}
\cline{2-3} Espaces Verts entretien Ville La Rochelle & 29\_04\_01\_01 & Espaces Verts entretien Ville La Rochelle\\
\hline
\end{supertabular}


\paragraph{Espaces Verts entretien Organisme HLM}
\noindent
\vspace{\baselineskip}

\renewcommand{\arraystretch}{1.2}
\begin{supertabular}{|p{10cm}|c|p{15cm}|}
\cline{2-3} Espaces Verts entretien Organisme HLM & 29\_04\_02\_01 & Espaces Verts entretien Organisme HLM\\
\hline
\end{supertabular}


\paragraph{Espaces Verts entretien Autre organisme ou privé}
\noindent
\vspace{\baselineskip}

\renewcommand{\arraystretch}{1.2}
\begin{supertabular}{|p{10cm}|c|p{15cm}|}
\cline{2-3} Espaces Verts entretien Autre organisme ou privé & 29\_04\_03\_01 & Espaces Verts entretien Autre organisme ou privé\\
\hline
\end{supertabular}
\subsection{Mobilier urbain}
\subsubsection{\large Stationnement}
\paragraph{Horodateur}
\noindent
\vspace{\baselineskip}

\renewcommand{\arraystretch}{1.2}
\begin{supertabular}{|p{10cm}|c|p{15cm}|}
\cline{2-3} Horodateur & 31\_01\_01\_01 & Horodateur\\
\hline
\end{supertabular}

\subsubsection{\large Transport}
\paragraph{Arceau stationnement vélo}
\noindent
\vspace{\baselineskip}

\renewcommand{\arraystretch}{1.2}
\begin{supertabular}{|p{10cm}|c|p{15cm}|}
\cline{2-3} Arceau stationnement vélo & 31\_03\_02\_01 & Arceaux stationnement vélo\\
\hline
\end{supertabular}


\paragraph{Arceau chicane}
\noindent
\vspace{\baselineskip}

\renewcommand{\arraystretch}{1.2}
\begin{supertabular}{|p{10cm}|c|p{15cm}|}
\cline{2-3} Arceau chicane & 31\_03\_03\_01 & Arceaux pour chicane\\
\hline
\end{supertabular}


\paragraph{Poteau d arrêt de bus}
\noindent
\vspace{\baselineskip}

\renewcommand{\arraystretch}{1.2}
\begin{supertabular}{|p{10cm}|c|p{15cm}|}
\cline{2-3} Poteau d arrêt de bus & 31\_03\_05\_01 & Poteau d arrêt de bus\\
\hline
\end{supertabular}


\paragraph{Borne Auto Plus}
\noindent
\vspace{\baselineskip}

\renewcommand{\arraystretch}{1.2}
\begin{supertabular}{|p{10cm}|c|p{15cm}|}
\cline{2-3} Borne Auto Plus & 31\_03\_06\_01 & Borne autoplus\\
\hline
\end{supertabular}

\subsubsection{\large Affichage}
\paragraph{Panneau d affichage libre}
\noindent
\vspace{\baselineskip}

\renewcommand{\arraystretch}{1.2}
\begin{supertabular}{|p{10cm}|c|p{15cm}|}
\cline{2-3} \multirow{2}{10cm}{Panneau d affichage libre} & 31\_04\_02\_01 & Panneau d affichage libre\\


\cline{2-3}                    & 31\_04\_02\_02 & Colonne d affichage libre\\
\hline
\end{supertabular}


\paragraph{Fourreau pour mat de pavoisement}
\noindent
\vspace{\baselineskip}

\renewcommand{\arraystretch}{1.2}
\begin{supertabular}{|p{10cm}|c|p{15cm}|}
\cline{2-3} \multirow{3}{10cm}{Fourreau pour mat de pavoisement} & 31\_04\_03\_01 & Fourreau pour Mat de Pavoisement\\


\cline{2-3}                    & 31\_04\_03\_02 & Drapeau\\


\cline{2-3}                    & 31\_04\_03\_03 & Pendule/horloge\\
\hline
\end{supertabular}

\subsubsection{\large Espaces Verts}
\paragraph{Objet Surfacique (vespachien, jeu d enfants, kiosque, … )}
\noindent
\vspace{\baselineskip}

\renewcommand{\arraystretch}{1.2}
\begin{supertabular}{|p{10cm}|c|p{15cm}|}
\cline{2-3} \multirow{4}{10cm}{Objet Surfacique (vespachien, jeu d enfants, kiosque, … )} & 31\_05\_01\_01 & Sanisette\\


\cline{2-3}                    & 31\_05\_01\_02 & Vespachien\\


\cline{2-3}                    & 31\_05\_01\_03 & Aire de Jeux d enfants\\


\cline{2-3}                    & 31\_05\_01\_04 & Kiosque\\
\hline
\end{supertabular}


\paragraph{Objet Linéaire ()}
\noindent
\vspace{\baselineskip}

\renewcommand{\arraystretch}{1.2}
\begin{supertabular}{|p{10cm}|c|p{15cm}|}
\cline{2-3} Objet Linéaire () & 31\_05\_02\_01 & Objet Linéaire Mobilier Urbain Espaces Verts\\
\hline
\end{supertabular}


\paragraph{Objet Ponctuel (Banc, Jardinière, … )}
\noindent
\vspace{\baselineskip}

\renewcommand{\arraystretch}{1.2}
\begin{supertabular}{|p{10cm}|c|p{15cm}|}
\cline{2-3} \multirow{2}{10cm}{Objet Ponctuel (Banc, Jardinière, … )} & 31\_05\_03\_02 & Jardinière\\


\cline{2-3}                    & 31\_05\_03\_03 & Grille d arbre\\
\hline
\end{supertabular}

\subsubsection{\large Sécurité}
\paragraph{Objet Surfacique (Poste de Contrôle d accès, … )}
\noindent
\vspace{\baselineskip}

\renewcommand{\arraystretch}{1.2}
\begin{supertabular}{|p{10cm}|c|p{15cm}|}
\cline{2-3} Objet Surfacique (Poste de Contrôle d accès, … ) & 31\_06\_01\_01 & Boucle capteur Poste de Contrôle d Accès\\
\hline
\end{supertabular}


\paragraph{Objet Linéaire (barrière de ville, chaîne, gabarit de hauteur, barrière levante, … )}
\noindent
\vspace{\baselineskip}

\renewcommand{\arraystretch}{1.2}
\begin{supertabular}{|p{10cm}|c|p{15cm}|}
\cline{2-3} \multirow{7}{10cm}{Objet Linéaire (barrière de ville, chaîne, gabarit de hauteur, barrière levante, … )} & 31\_06\_02\_01 & Barrière fixe - Croix de saint andré\\


\cline{2-3}                    & 31\_06\_02\_02 & Chaîne\\


\cline{2-3}                    & 31\_06\_02\_03 & Gabarit de hauteur\\


\cline{2-3}                    & 31\_06\_02\_04 & Barrière levante\\


\cline{2-3}                    & 31\_06\_02\_05 & Obstacle de passage\\


\cline{2-3}                    & 31\_06\_02\_06 & Glissière de sécurité\\


\cline{2-3}                    & 31\_06\_02\_07 & Parapet\\
\hline
\end{supertabular}


\paragraph{Objet Ponctuel (Borne / Potelet, … )}
\noindent
\vspace{\baselineskip}

\renewcommand{\arraystretch}{1.2}
\begin{supertabular}{|p{10cm}|c|p{15cm}|}
\cline{2-3} \multirow{4}{10cm}{Objet Ponctuel (Borne / Potelet, … )} & 31\_06\_03\_01 & Borne fixe / potelet\\


\cline{2-3}                    & 31\_06\_03\_02 & Borne escamotable\\


\cline{2-3}                    & 31\_06\_03\_03 & Poteau lecteur Poste de Contrôle d Accès\\


\cline{2-3}                    & 31\_06\_03\_04 & Caméra\\
\hline
\end{supertabular}

\subsubsection{\large Services (souvent externe à la ville de La Rochelle)}
\paragraph{Boite aux lettres}
\noindent
\vspace{\baselineskip}

\renewcommand{\arraystretch}{1.2}
\begin{supertabular}{|p{10cm}|c|p{15cm}|}
\cline{2-3} Boite aux lettres & 31\_07\_01\_01 & Boite aux lettres\\
\hline
\end{supertabular}


\paragraph{Cabine téléphonique}
\noindent
\vspace{\baselineskip}

\renewcommand{\arraystretch}{1.2}
\begin{supertabular}{|p{10cm}|c|p{15cm}|}
\cline{2-3} Cabine téléphonique & 31\_07\_02\_01 & Cabine téléphonique\\
\hline
\end{supertabular}


\paragraph{Coffret relais de la poste}
\noindent
\vspace{\baselineskip}

\renewcommand{\arraystretch}{1.2}
\begin{supertabular}{|p{10cm}|c|p{15cm}|}
\cline{2-3} Coffret relais de la poste & 31\_07\_03\_01 & Coffret relais de La Poste\\
\hline
\end{supertabular}


\paragraph{Container déchet}
\noindent
\vspace{\baselineskip}

\renewcommand{\arraystretch}{1.2}
\begin{supertabular}{|p{10cm}|c|p{15cm}|}
\cline{2-3} Container déchet & 31\_07\_04\_01 & Container à déchet\\
\hline
\end{supertabular}
\subsection{Geodesie, Points de référence et Habillage}
\subsubsection{\large Géodésie}
\paragraph{Point géodésique}
\noindent
\vspace{\baselineskip}

\renewcommand{\arraystretch}{1.2}
\begin{supertabular}{|p{10cm}|c|p{15cm}|}
\cline{2-3} Point géodésique & 33\_01\_01\_02 & Station de polygonation ou station GPS\\
\hline
\end{supertabular}


\paragraph{Point nivellement}
\noindent
\vspace{\baselineskip}

\renewcommand{\arraystretch}{1.2}
\begin{supertabular}{|p{10cm}|c|p{15cm}|}
\cline{2-3} \multirow{3}{10cm}{Point nivellement} & 33\_01\_02\_01 & Repère de nivellement\\


\cline{2-3}                    & 33\_01\_02\_02 & Point altimétrique\\


\cline{2-3}                    & 33\_01\_02\_03 & Texte altitude du point\\
\hline
\end{supertabular}


\paragraph{Courbe de niveau}
\noindent
\vspace{\baselineskip}

\renewcommand{\arraystretch}{1.2}
\begin{supertabular}{|p{10cm}|c|p{15cm}|}
\cline{2-3} \multirow{2}{10cm}{Courbe de niveau} & 33\_01\_03\_01 & Courbe de niveau\\


\cline{2-3}                    & 33\_01\_03\_02 & Texte altitude de la courbe\\
\hline
\end{supertabular}

\subsubsection{\large Points de référence}
\paragraph{Position géographique d un appareil de fontainerie du réseau eau potable ou gaz}
\noindent
\vspace{\baselineskip}

\renewcommand{\arraystretch}{1.2}
\begin{supertabular}{|p{10cm}|c|p{15cm}|}
\cline{2-3} \multirow{3}{10cm}{Position géographique d un appareil de fontainerie du réseau eau potable ou gaz} & 33\_02\_01\_01 & Bouche a clef eau potable (position levée par géomètre)\\


\cline{2-3}                    & 33\_02\_01\_02 & Bouche à clef gaz (position levée par géomètre)\\


\cline{2-3}                    & 33\_02\_01\_03 & Poteau incendie (position levée par géomètre)\\
\hline
\end{supertabular}


\paragraph{Position géographique d un regard ou d une plaque}
\noindent
\vspace{\baselineskip}

\renewcommand{\arraystretch}{1.2}
\begin{supertabular}{|p{10cm}|c|p{15cm}|}
\cline{2-3} \multirow{7}{10cm}{Position géographique d un regard ou d une plaque} & 33\_02\_02\_01 & Position géographique d un regard rond\\


\cline{2-3}                    & 34\_02\_02\_02 & Position géographique d un regard carré\\


\cline{2-3}                    & 35\_02\_02\_03 & Position géographique d une plaque ronde\\


\cline{2-3}                    & 36\_02\_02\_04 & Position géographique d une plaque simple\\


\cline{2-3}                    & 37\_02\_02\_05 & Position géographique d une plaque double\\


\cline{2-3}                    & 38\_02\_02\_06 & Position géographique d une plaque triple\\


\cline{2-3}                    & 39\_02\_02\_07 & Position géographique d une plaque quadruple\\
\hline
\end{supertabular}


\paragraph{Position géographique d un Support (système d accrochage vertical)}
\noindent
\vspace{\baselineskip}

\renewcommand{\arraystretch}{1.2}
\begin{supertabular}{|p{10cm}|c|p{15cm}|}
\cline{2-3} Position géographique d un Support (système d accrochage vertical) & 33\_02\_03\_01 & Position géographique d un Support (système d accrochage vertical)\\
\hline
\end{supertabular}

\subsubsection{\large Habillage}
\paragraph{Élément d habillage non géo-référencé}
\noindent
\vspace{\baselineskip}

\renewcommand{\arraystretch}{1.2}
\begin{supertabular}{|p{10cm}|c|p{15cm}|}
\cline{2-3} \multirow{4}{10cm}{Élément d habillage non géo-référencé} & 33\_03\_01\_01 & Cadre et cartouche\\


\cline{2-3}                    & 33\_03\_01\_02 & Logo\\


\cline{2-3}                    & 33\_03\_01\_03 & Flèche nord\\


\cline{2-3}                    & 33\_03\_01\_04 & Barre d échelle graphique\\
\hline
\end{supertabular}


\paragraph{Élément d habillage géo-référencé}
\noindent
\vspace{\baselineskip}

\renewcommand{\arraystretch}{1.2}
\begin{supertabular}{|p{10cm}|c|p{15cm}|}
\cline{2-3} Élément d habillage géo-référencé & 33\_03\_02\_03 & Emprise géographique d une opération\\
\hline
\end{supertabular}


\paragraph{Texte divers non géo-référencé}
\noindent
\vspace{\baselineskip}

\renewcommand{\arraystretch}{1.2}
\begin{supertabular}{|p{10cm}|c|p{15cm}|}
\cline{2-3} \multirow{5}{10cm}{Texte divers non géo-référencé} & 33\_03\_03\_01 & Date\\


\cline{2-3}                    & 33\_03\_03\_02 & Auteur\\


\cline{2-3}                    & 33\_03\_03\_03 & Échelle du plan d origine\\


\cline{2-3}                    & 33\_03\_03\_04 & Légende\\


\cline{2-3}                    & 33\_03\_03\_05 & Autre Texte du Cartouche\\
\hline
\end{supertabular}


\paragraph{Texte divers géo-référencé}
\noindent
\vspace{\baselineskip}

\renewcommand{\arraystretch}{1.2}
\begin{supertabular}{|p{10cm}|c|p{15cm}|}
\cline{2-3} \multirow{6}{10cm}{Texte divers géo-référencé} & 33\_03\_04\_01 & Nom du quartier\\


\cline{2-3}                    & 33\_03\_04\_02 & Nom de la voie\\


\cline{2-3}                    & 33\_03\_04\_03 & Numéro du bâtiment (adresse postale / numéro de façade)\\


\cline{2-3}                    & 33\_03\_04\_04 & Nom du bâtiment\\


\cline{2-3}                    & 33\_03\_04\_05 & Nom de la société occupant le bâtiment\\


\cline{2-3}                    & 33\_03\_04\_06 & Nombre détage pour un bâtiment (ex: R+2)\\
\hline
\end{supertabular}

\end{document}