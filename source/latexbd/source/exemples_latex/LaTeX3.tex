% Tout ce qui est mis derri�re un � % � n'est pas vu par LaTeX
% On appelle cela des � commentaires �.  Les commentaires permettent de
% commenter son document - comme ce que je suis en train de faire
% actuellement - et de cacher du code - cf. la ligne \pagestyle.

\documentclass[a4paper]{article}

\usepackage[latin1]{inputenc}
\usepackage[cyr]{aeguill}
\usepackage{xspace}
\usepackage{amsmath}
\usepackage[english,francais]{babel}
\usepackage{url}
\let\urlorig\url
\renewcommand{\url}[1]{%
  \begin{otherlanguage}{english}\urlorig{#1}\end{otherlanguage}%
}
\renewcommand{\contentsname}{Sommaire} % si tableofcontents au d�but
\newcommand{\Numero}{\No}
\newcommand{\numero}{\no}
\newcommand{\fup}[1]{\up{#1}}
%%% N'oubliez pas les espaces devant les doubles ponctuations
\NoAutoSpaceBeforeFDP


\title{Mon premier document Latex}   % Les param�tres du titre : titre, auteur, date
\author{Fr�d�ric Garel}
\date{}                       % La date n'est pas requise (la date du
                              % jour de compilation est utilis�e en son
			                        % absence

\begin{document}

\maketitle                    % Faire un titre utilisant les donn�es
                              % pass�es � \title, \author et \date

\begin{abstract}
  R�sum� r�sum� r�sum�, etc.  % R�sum� de l'article
\end{abstract}

% \tableofcontents            % Table des mati�res

% \listoffigures              % Table des figures

% \listoftables               % Liste des tableaux

\part{Titre de la 1ere Partie}                  % Commencer une partie...

\section{Titre de la section}               % Commencer une section, etc.

\subsection{Titre de la sub section}            % Section plus petite

\subsubsection{Titre de la sub sub section}         % Encore plus petite

\paragraph{Titre du paragraphe}             % Toutes petites sections (le nom \paragraph
                              % n'est pas tr�s bien choisi)

\subparagraph{Titre du sous-paragraphe}          % La derni�re

\appendix                     % Commen�ons les annexes

\section{Titre de l'annexe A}               % Annexe A

\section{Titre de l'annexe B}               % Annexe B

\end{document}